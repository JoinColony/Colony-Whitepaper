% \section{Voting}\label{sec:voting}
% Voting is a necessity in the Colony Network. In a smoothly running colony, votes will not be needed, but once there is a disagreement, or there is the need to change something that will affect a large number of people, a poll will be needed to gauge the opinion of those in a domain or even whole colony.

% When a poll is required, the following are defined
% \begin{equation*}
%   \ascode{poll\_id} \rightarrow 
%   \begin{cases}
%     \ascode{consequence} &	\textnormal{Contains what will happen if `yes' wins.}\\
%     \ascode{options[]} &	\parbox[t]{.6\linewidth}{\textnormal{Yes, and No.}}\\
%     \ascode{start\_time} &	\textnormal{time when the vote opens.}\\
%     \ascode{close\_time} &	\textnormal{time when the vote closes.}\\
%     \ascode{status} &	\parbox[t]{.6\linewidth}{\textnormal{One of created, active (open for votes), active (revealing votes) or resolved.}}
%   \end{cases}
% \end{equation*}

% \ascode{poll\_id} is unique only within a colony. The \ascode{consequence} entry is a structure formatted so that it can be used in a on-chain to implement the proposed changes should the `Yes' option win.\footnote{The exact structure of this is dependent on the method used to implement contract upgradeability. The function that uses it is likely to require being coded with inline assembly in the contracts, and require significant effort in the client to make it intuitive to generate and verify.}

% The duration of the poll is determined by the amount of reputation eligible to vote in the poll as a fraction of reputation in the colony. If a larger fraction is eligible, the longer the poll is open for. The minimum duration is two days and the maximum is seven. In the event of an exclusively token-weighted vote, the duration is seven days.

% Voting takes place using a commit-and-reveal-scheme. To make a vote, the user submits a hash that is \ascode{keccak(secret, vote\_id)}, where \ascode{option\_id} indicates the option that the user is voting for. Once voting has closed, the poll enters the reveal phase, where a user can submit \ascode{secret, option\_id} and the contract calculates \ascode{keccak(secret, option\_id)} to verify it is what they originally submitted.

% As the secret is revealed it cannot be sensitive. It must also change with each vote so that observers cannot establish what people are voting for after they have revealed their first vote. We suggest a (hash) of the consequence field of the poll signed with their private key. This is easily reproducible by a client at a later date with no local storage required.

% The exact consequences of revealing a vote depend on the type of vote being conducted; these types are described in the next section.

% Once a vote has been in the reveal phase for 48 hours, a transaction may be made to finalise the vote. Any subsequent reveals of votes do not contribute to the decision being made, but serve only to unlock the user's tokens if it was a token-weighted or hybrid vote (see below). We define quorum to be more than 10\% of the reputation eligible to vote has done so. If quorum is not reached, no changes are made and all participants get their remaining staked tokens returned.\footnote{Some of the staked tokens on the change side will have been used to compensate voters.}


% \subsection{Types of vote}
% \subsubsection{Reputation weighted voting}
% Most votes in a colony will be due to disputes (see Section \ref{sec:disputes}). In these cases, the weights of the users' votes is proportional to the reputation that each user has in the domain and skill that the vote is taking place in. When such a vote starts, the current reputation state is stored alongside the vote. This allows the current reputation state to be `frozen' for the context of the vote, and prevents unwanted behaviours that might otherwise be encouraged (for example, delaying submission of a task until closer to voting so that the reputation earned has not decayed as much).


% When revealing their vote, the user also supplies a Merkle proof of their relevant reputation contained within the reputation state that was saved at the start of the vote. The total vote for the option they demonstrated they voted for is then incremented appropriately.

% \subsubsection{Token weighted voting}
% Unlike with reputation, we do not have the ability to `freeze' the token distribution when a vote starts. While this is effectively possible with something like the MiniMe token \cite{minime}, we envision token-weighted votes will still be regular enough within a Colony that we do not wish to burden users with the gas costs of deploying a new contract every time.

% When conducting a token weighted vote, steps must be taken to ensure that tokens cannot be used to vote multiple times. In the case of ``The DAO'', once a user had voted their tokens were locked until the vote completed. This introduced peculiar incentives to delay voting until as late as possible to avoid locking tokens unnecessarily.  Our locking scheme avoids such skewed incentives.

% Instead, once a vote enters the reveal phase, any user who has voted on that vote will find themselves unable to see tokens sent to them, or be able to send tokens themselves --- their token balance has become locked. To unlock their token balance, users only have to reveal the vote they cast for any polls that have entered the reveal phase --- something they can do at any time. Once their tokens are unlocked, any tokens they have notionally received since their tokens became locked are added to their balance.

% It is possible to achieve this locking in constant gas by storing all submitted secrets for votes in a sorted linked list indexed by \ascode{close\_time}. If the first key in this linked list is earlier than \ascode{now} when a user sends or would receive funds, then they find their tokens locked. Revealing a vote causes the key to be deleted (if the user has no other votes submitted for polls that closed at the same time). This will unlock the tokens so long as the next key in the list is a timestamp in the future. A more detailed description of once such implementation can be found on the Colony blog \cite{ColonyVoting}.

% Insertion into this structure can also be done in constant gas if the client supplies the correct insertion location, which can be checked efficiently on-chain, rather than searching for the correct location to insert new items.

% \subsubsection{Hybrid voting}
% A hybrid vote would allow both reputation holders and token holders to vote on a decision. In order to pass such a vote, a proposal must receive support from a majority of both tokens and reputation. We envision such a vote being used when the action being voted on would potentially have a sizeable impact on both reputation holders and token holders. This would include altering the supply of the colony tokens beyond the parameters already agreed (see Section \ref{sec:colony-token-management}) or when deciding whether to execute an arbitrary transaction (see Section \ref{sec:arbitrary-transaction}).

% In order for a proposal to successfully pass through a hybrid vote, both the reputation holders and the token holders must have reached quorum, and a majority of both reputation and token holders who vote must agree that the change should be enacted.
\section{The Colonies' own Tokens}\label{sec:colony-tokens}
Every colony has its own token. These are the tokens that, when earned as a task bounty, also create reputation for the receiver. What these tokens represent apart from this is up to the colony to establish. For example, they may have financial value, or they may be purely symbolic; some possible scenarios are outlined in Section \ref{sec:colony-token-examples}.

\subsection{Managing a colony's token supply}\label{sec:colony-token-management}
A colony is in control of changing the supply of its own tokens. More specifically, both reputation holders and token holders must agree to changes in the token supply, as both will be affected by it.

\subsubsection{Token Generation and Initial Supply}
When a colony is created, the \ascode{TokenSupplyCeiling} and the \ascode{TokenIssuanceRate} are set. The former is the total number of colony tokens that will be created and the latter is the rate at which they become available to the colony-wide domain to assign to tasks or subdomains. The number of tokens available to the colony-wide domain can be updated at any time by a transaction from any user.

At colony creation, some tokens must also be assigned to addresses to allow users to stake tokens to create the first tasks. A one-off lump sum may also be created and made available to the colony-wide domain.

\subsubsection{Increasing the TokenSupplyCeiling}
 It is crucial that new tokens cannot be generated without widespread consensus --- especially if tokens have a financial value. Consequently, such decisions require a vote with high quorum and majority requirements involving both the token holders and reputation holders.

\subsubsection{Changing the TokenIssuanceRate}
The \ascode{TokenSupplyCeiling} represents the tokens that the token holders have granted to the colony in order to conduct business: to fund tasks and domains, and to hire workers and contributors. This is especially important during the early life of a colony when it has little-to-no revenue in other tokens to fall back on.

The \ascode{TokenIssuanceRate} controls how rapidly the colony receives the new tokens. If the rate is `too high', tokens will accumulate in the pot of the top-level colony domain (or other pots lower in the hierarchy); usually this is not a big problem. If the rate is too low, this signals that the colony has a healthy amount of activity and that the issuance rate has become a bottleneck. In such situations it may be desirable to increase the rate of issuance without necessarily increasing the maximum supply.

Increasing and decreasing the \ascode{TokenIssuanceRate} by up to 10\% can be done by the reputation holders alone and this action can be taken no more than once every 4 weeks. Larger changes to the issuance rate should additionally require the agreement of existing token holders.


\subsection{Example of Token Usage}\label{sec:colony-token-examples}
\subsubsection*{Tokens as early rewards}
One of the chief benefits of a colony having its own token is that it can offer rewards for work before it has any revenue or external funding to draw on.
A new colony may offer token bounties for tasks that people may accept in the hope that the reputation earned by these token payments and the future revenue earned by the colony will eventually reap financial rewards. By allowing `spending' before fund-raising, the financial burden during the start-up phase of a new colony is eased. Once a colony is profitable, payment in tokens may be the exception rather than the norm.

\subsubsection*{Tokens representing hours worked}
We could imagine a colony in which all tasks are paid in Ether, but include a number of the colony's own tokens as well, equal to the expected number of hours worked on a task. The members of the colony would be responsible for assigning `correct' token and Ether bounties to tasks. This extra responsibility would also ensure users doing the same amount of work received the same reputation gain, rather than the reputation gain being dependent on the rates they charged.

%\subsubsection*{Tokens as `fake internet points'}
%Tokens themselves need not have a monetary value; (a non-profit Colony for instance may choose to continually generate new tokens and make them available freely to anyone via a `faucet contract' in order to \emph{ensure} that they do not have a direct value). In such a situation, the token would only be valuable in a context in which payout also confers reputation.

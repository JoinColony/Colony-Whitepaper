

\section{Special Actions and Example Configurations}\label{sec:special-cases}

\subsection{Proposing an arbitrary transaction by the Colony contract}\label{sec:arbitrary-transaction}
It is desirable to have a mechanism by which a colony can create an arbitrary transaction on the blockchain to interact with contracts and tokens without requiring the network to explicitly support them. Such transactions should be rare occurrences with high threshold requirements.

Formally, proposing that a colony make an arbitrary transaction on the blockchain is no different from an objection; however the proposal is to change the value of a special variable from zero to the value of the transaction data of the proposed transaction. Such a proposal requires the entire colony to be able to vote (both token holders and reputation holders), as it concerns actions taken `by the contract as a whole'. In the event the proposal is successful, the special variable is set. Another subsequent transaction --- able to be made by anyone --- is able to call a function that executes the transaction in the special variable, and resets it to zero if successful.


\subsection{Emergency Shutdown}\label{sec:big-red-button}
As indicated in section \ref{sec:escape-hatches}, a transaction from a whitelisted address can put a colony into `recovery mode' during which the state can be edited, the effects of bugs can be corrected and upgrades can be pushed. In addition to being able to stop-and-repair colonies, the reputation mining process will also have an emergency stop-and-repair mechanism (at least initially). This will allow the whitelisted addresses to revert the reputation root hash to a previous version and halt all updates to the reputation state until the issues have been resolved. The colonies will be able to continue operations as usual using the reputations of the last valid state.

\subsection{Example Configurations}\label{sec:example-configs}
The Colony Governance Protocol described in this document is concerned with what happens `on chain', i.e. those actions that directly affect the Ethereum blockchain. Users of the network however are not expected to ever interact with the contracts manually; instead they will be using some front-end application that makes all of the network's functionality intuitive and simple.

In any colony application we expect a certain amount of \textbf{front-end abstraction} in which complex tools and concepts are presented for the users' convenience, and translated in the background into a sequence of contract interactions on the colony network.

Front-end abstraction lets us realise certain functionality that doesn't \emph{seem} to be part of our protocol but can actually be coded into the governance system and made usable by a well designed application. The remainder of this section describes some such cases.
%

\subsubsection{Salaried Positions}\label{sec:salary}

The work-for-payment model in the colony network is based around tasks, and on the surface this implies colony-worker relationships that are purely transactional. However the system is flexible enough to accommodate a wider range of employment models. One such example is a \emph{salaried position}.

A salaried position could be realised by creating a special domain representing the position to be filled. The domain could be issued the salary through a priority funding proposal. The employee would be the only person with reputation within that domain and would be able to withdraw funds by creating and self-assigning placeholder tasks that are funded from the domain's pot. A good frontend could hide these implementation details from the users and render salaried positions differently from regular domains.

\subsubsection{Awarding Reputation Bonuses}

All reputation decays, as described in Section \ref{sec:reputation}. This prevents an eternal `reputation aristocracy' and allows reputation to be meaningful even after major changes in the colony token's value. 

Reputation is awarded when a user receives payment of colony tokens --- most commonly as payout from a task, but sometimes from dispute resolution and, in the case of the \rc, from the reputation mining process. We can use the task payout mechanism to award users extra reputation provided there is consensus to do so. 

Consider the scenario in which a founder, or an early important contributor to a colony has almost no reputation left by the time the colony starts earning revenue; perhaps the development of the product took a long time or perhaps the reputation decay rate was sub-optimally high for the particular colony\footnote{Finding an optimal decay rate for reputation in the network will depend on empirical data collected from early colonies.}. To get around the limitations of the reputation system and to re-integrate the founder (and make them eligible to receive their rewards), the colony can create a special `fake' task that is solely designed to award reputation. To qualify for the payout of tokens (and thereby the reputation), the user in question would have to give the same number of tokens back to the colony. Again, a good frontend abstraction could make such reputation awards easy and intuitive.

The important point is that any limitations imposed by the system can be weakened \emph{if there is consensus to do so}. The system should not stand in the way of consensus, it should just provide conflict resolution mechanisms for those times in which there is dissent.


\subsubsection{Objections by non-members}

Having reputation is a prerequisite for creating an objection and triggering the dispute process. Therefore, if an outsider is hired by a colony to perform a task, they will not, on their own, be able to object to the evaluation of their work. However, a good colony frontend may allow them to create the template for an objection, effectively calling for members of the colony to support it and submit the objection to the colony network on-chain on their behalf.  

This is analogous to a member staking only 10\% of the required amount and waiting for further support from their peers (Section \ref{sec:costs-of-disputes}), with the difference being that without any third party support, this `objection' would never be processed on-chain.
\section{Special Actions and Example Configurations}\label{sec:special-cases}


\subsection{Proposing an arbitrary transaction by the Colony contract}\label{sec:arbitrary-transaction}
It is desirable to have a mechanism by which a colony can create an arbitrary transaction on the blockchain to interact with contracts and tokens beyond those whitelisted by the network in advance. Such transactions should be rare occurrences with high threshhold requirements.

Formally, proposing that a colony make an arbitrary transaction on the blockchain is no different from an objection; the proposal is to change the value of a special variable from zero to the value of the transaction data of the proposed transaction.\\
Such a proposal requires the entire colony to be able to vote (possibly both token holders and/or reputation holders), as it concerns actions taken `by the contract as a whole'. In the event the proposal is successful, the special variable is set. Another subsequent transaction - able to be made by anyone - is able to call a function that executes the transaction in the special variable, and resets it to zero if successful.

\subsection{Token-weighted, reputation-weighted and hybrid voting}
The majority of decisions in a colony are purely reputation weighted (even though creating a vote requires stake of both tokens and reputation), though there is no reason why a more traditional token-weighted vote shouldn't be available for some decisions, nor a hybrid vote based on both reputation and token holdings. In both of these cases, every account is allowed to vote, and in the case of a hybrid vote, all reputation is eligible to vote. When a hybrid vote takes place, the total reputation and the total token holdings each represent 50\% of the voting weight.

The primary use of a token weighted vote is related to the management of the colony tokens itself; it seems reasonable that the decision and ability to create more tokens should lie with the colony token holders.

 
\subsection{Managing a colony's token supply}\label{sec:colony-token-managements}
\subsubsection{Initial Supply}
When a colony is created, the \ascode{TotalTokenSupply} and the \ascode{TokenIssaunceRate} are set. The former is the total number of colony tokens that will be created and the latter is the rate at which they become available to the colony-wide domain. Just as with any funding proposal (Section \ref{sec:finance}), users can issue a `ping' to update the totals available to the domain.

\subsubsection{Increasing the Token Supply}
In order to increase the \ascode{TotalTokenSupply} and thereby allow for new tokens to be generated, the following events must occur
\begin{itemize}
 \item A proposal is made to increase the value of \ascode{TotalTokenSupply}
 \item The proposal is backed by 10\% of all colony reputation
 \item A colony wide vote of tokens and reputation is held
 \item A majority of reputation voted in favour of the increase
 \item A two-thirds supermajority of tokens voted in favour
 \item Over 30\% of tokens participated in the vote.
\end{itemize}

\subsubsection{Changing the TokenIssaunceRate}
The \ascode{TotalTokenSupply} represents the tokens that the token holders have granted to the reputation holders in order to conduct business: to fund tasks and domains, to hire workers and contributors; especially during the early life of a colony in which it has litte to no revenue in other tokens to fall back on.\\
The \ascode{TokenIssaunceRate} is a measure of how rapidly the colony is able to absorb and allocate the new tokens. If the rate is `too high', tokens will accumulate in the colony wide domain's pot (or other pots lower in the hierarchy); usually this is not a big problem. If the rate is too low however, this signals that the colony has a healthy amount of activity and that the issuance rate has become a bottleneck. In such situations it may be desirable to increase the rate of issuance without necessarily increasing the maximum supply. \\
Increasing and decreasing the \ascode{TokenIssaunceRate} by up to 10\% can be done by the reputation holders alone and this action can be taken no more than once every 4 weeks. Bigger changes to the issuance rate require a double majority of tokens and reputation.

\subsection{Emergency Shutdown}
%
Placeholder\\
How to trigger an emergency shutdown -- the "big red button" we talked about.
Conditions for restarting
%

\subsection{Example Configurations}\label{sec:example-configs}
The Colony Governance Protocol described in this whitepaper is concerned with what happens `on chain', i.e. those actions that directly affect the ethereum blockchain. Users of the network however are not expected to ever interact with the contracts manually; instead they will be using some front-end application that makes all of the network's functionality appear intuitive and simple.\\
In any colony application we expect a certain amount of \textbf{front-end abstraction} in which complex tools and concepts are presented for the users' convenience, and translated in the backgrounnd into a sequence of contract interactions on the colony network.

Front-end abstraction lets us realise certain functionality that doesn't \emph{seem} to be part of our protocol but can actually be coded into the governance system and made usable by a well designed application. Some examples follow.
%

\subsubsection{Salaried Positions}\label{sec:salary}

The work-for-payment model in the colony network is based around tasks, and you'd be forgiven for thinking that this implied colony-worker relationships that are purely transactional. However the system is in fact flexible enough to accomodate a wider range of employment models. One such example is a \emph{salaried position}.

A slaried position could be realised by creating a special domain representing the position to be filled. The domain could be issued the salary through a mandated funding proposal. The hiree would be the only person with reputation within that domain and would be able to withdraw funds by creating and self-assigning placeholder tasks that are funded from the domain's pot. A good frontend could hide these implementation details from the users and render salaried positions differently from regular domains.

\subsubsection{Awarding Reputation Bonuses}

All reputation decays, as described in section \ref{sec:reputation}. This prevents an eternal `reputation aristocracy' and allows reputation to be meaningful even after major changes in the colony token's value. \\
Reputation is awarded when a user receives payment of colony tokens - usually as payout from a task. We can use this mechanim to award users extra reputation provided there is consensus to do so. 

Consider the scenario in which a founder, or an early important contributor to a colony has almost no reputation left by the time the colony starts earning revenue; perhaps the development of the product took a long time or perhaps the reputation decay rate was suboptimally high for the particular colony\footnote{Finding an optimal decay rate for reputation in the network will depend on empirical data collected from early colonies. As with similar constants in this paper, the numerical values may change before the full network is live.}. To get around the limitations of the reputation system and to re-integrate the founder (and make them eligible to receive their just rewards), the colony can create a special `fake' task that is solely designed to award reputation. To qualify for the payout of tokens (and thereby the reputation), the user in question would have to give the same number of tokens back to the colony. Again, a good frontend abstraction could make such reputation awards easy and intuitive.

The important point is that any limitations imposed by the system can be weakened \emph{if there is consensus to do so}. The system should not stand in the way of consensus, it should just provide conflict resolution mechanisms for those times in which there is dissent.


\subsubsection{Objections by non-members}
%
Placeholder\\
explain how without reputation you cannot actually launch an objection, but describe how the frontend can make it easy for the user to propose an objection and ask rep-holders to join. This is analogous to users staking only 10\% of the requirement and waiting for other backers.
%

\subsubsection{Firing members of a colony}

%
Placeholder\\
there isn't really a concept of firing per se, but what we mean here is a special action that sets all of a users reputation to zero.
Threshold, 2/3 of all rep voting in favour, quorum at least 10\%. 
%

Did we want to keep this functionality? I don't remember.


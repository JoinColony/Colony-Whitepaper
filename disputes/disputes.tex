\section{Objections and Disputes}\label{sec:objections-and-disputes}\label{sec:disputes}
The most successful organisations are those which are able to effectively and efficiently make decisions, divide labour, and deploy resources. These are trust problems, which have traditionally been solved by management hierarchies. But colonies are intended to be low trust, decentralised, and pseudonymous --- a hierarchy is not suitable.

The solution to collective decision making is usually voting, but Colony is designed for day to day operation of an organisation. Voting on every decision is wholly impractical.

The emphasis should be on `getting stuff done' and not about `applying for permission'. Therefore, Colony is designed to be permissive. Task creation does not require explicit approval (Section \ref{sec:tasks}), nor do basic funding proposals (Section \ref{sec:finance}) or any number of administrative actions throughout the Colony system.

The \textbf{Dispute System} provides a self regulating mechanism to provide a balanced set of incentives for users to keep their colony running harmoniously. It is there to resolve disagreements and to punish bad behaviour and fraud. The dispute mechanism allows colony members to signal disapproval and potentially \textbf{force a vote} on decisions and actions that would otherwise have proceeded unimpeded.

\subsubsection*{What are objections?}
When a member of a colony feels that something is amiss, they can \emph{raise an objection}. By doing so, they are fundamentally proposing that a variable, or more than one variable, in the contract should be changed to another value. For this reason we call supporters of the objection the `change' side and opponents the `keep' side.

The user raising the objection must also put up a stake of colony tokens to back it up (see Section \ref{sec:costs-of-disputes}). In essence, they are challenging the rest of the colony to disagree with them. In the spirit of avoiding unnecessary voting, the objection will pass automatically \emph{unless} someone else stakes on the `keep' side and thereby elevates the objection to a \emph{dispute}.

\subsubsection*{What are disputes?}
We say that a dispute has been raised whenever an objection has found enough support on both the `change' side as well as the `keep' side. Once raised, disputes must be resolved by voting.

\subsection{Raising objections}\label{subsec:raising-objections}

The user raising an objection submits the following data:
\begin{itemize}
 \item the data that should be changed
 \item the reputation(s) that should vote on this issue (a maximum of one from each of the domain and skill hierarchies)
 \item proof that these reputations should be allowed to make the change in question.
\end{itemize}

The first item identifies the subject of the objection, and what the initiator believes the state should be.\footnote{The exact structure of this is dependent on the method used to implement contract upgradability. The function that uses it is likely to require being coded with inline assembly in the contracts, and require significant effort in the client to make it intuitive to generate and verify.} The second and third points concern \emph{escalation}.

\begin{center}
 \textbf{In Colony you cannot escalate a decision to higher management, you can only escalate to bigger groups of your peers.}
\end{center}

For example, suppose that the objection concerns a task in the domain `development of our website'. The objector could choose to have all `development' reputation vote on it --- we say the decision was `escalated to the development domain'. In this example, the third point would be a proof that the domain `development of our website' was indeed a subdomain of `development'. The highest domain any decision can be escalated to is that of the entire colony, where all domain reputation is entitled to vote.

Whenever an escalation occurs, we need to ensure that the reputation we are escalating to is a direct parent of the reputation associated with the variable being changed. This is possible to do efficiently because of metadata that is placed on the reputations (for domains) when they are created, which includes pointers to at least the direct parent.\footnote{Each reputation type includes the \ascode{parents} array, containing pointers to its `$n^{th}$ parents' for all $n$ that are powers of $2$ (see Section \ref{subsec:on-chain-representation-of-skills}).} When a user creates an objection, instead of directly specifying the domain they are escalating to, they provide the steps needed to get there from the domain associated with the variable that is to be changed. This ensures that the domain they escalate is a direct parent of that associated with the variable.

\subsection{Costs and rewards}\label{sec:costs-of-disputes}
\subsubsection{Cost of raising an objection}
To create an objection, a user must possess enough reputation and must also stake some number of the colony's tokens. The reputation they need to be able to make the objection depends on the level they are escalating to; the `higher up' the decision goes, the higher the reputation requirement. To be able to create an objection, the user must have 0.1\% of the reputation queried and then must stake 0.1\% of the corresponding fraction of tokens. Thus, if an objection appeals to 13\% of total colony reputation, then objecting requires 0.013\% (0.1\% of 13\%) of reputation and the required stake is 0.013\% of all colony tokens.

If the initial user does not have the required number of tokens or reputation, they can still create such a proposal by staking as little as 10\% of the tokens required, which requires them to have a correspondingly smaller amount of reputation.\footnote{This minimum amount required to even propose a change prevents users from spamming objections --- even those that won’t ever be voted on --- to large numbers of people, which would impede the smooth running of the colony.} In this case the objection will not be `live' until other users stake tokens, and take it over the 0.1\% threshold. The  amount of tokens required to be staked for a particular objection is recorded at the time when it is created. Users can only stake tokens in proportion to the reputation they have. For example, if they wanted to stake 40\% of the tokens required, they must have at least 40\% of the reputation that would be required to create the objection outright.

\subsubsection{Cost of defending against a raised objection}
Once enough tokens have been staked against an objection it becomes active and, barring any further user actions for three days, the suggested change will take place when the objection is `pinged' by a user. However, if there are users who oppose the suggested `change', they may stake tokens in support of the `keep' side. If the keep side receives sufficient support, a dispute is raised.

If the `change' side does not garner enough support in three days, the objection fails and is rejected. If, three days after the `change' side had enough tokens staked and the `keep' side does not, then it is assumed that the change is acceptable and it occurs when `pinged'.

\subsubsection{Voting on disputes}
If both sides stake the required number of tokens within their three days time limit, then the proposal goes to a vote. The weight of a user's vote is the sum of their reputations in the skills chosen by the user who originally raised the objection.

The duration of the poll is determined by the amount of reputation eligible to vote in the poll as a fraction of reputation in the colony. If a larger fraction is eligible, the longer the poll is open for. The minimum duration is two days and the maximum is seven. This is a trade-off between allowing disagreements between small groups to be resolved quickly, but to also allow adequate debate to occur when more people are involved.

Voting takes place using a commit-and-reveal-scheme. To make a vote, the user submits a hash that is \ascode{keccak256(secret, option\_id)}, where \ascode{option\_id} indicates the option that the user is voting for. Once voting has closed, the poll enters the reveal phase, where a user can submit \ascode{secret, option\_id} and the contract calculates \ascode{keccak256(secret, option\_id)} to verify it is what they originally submitted.

As the secret is revealed it cannot be sensitive. It must also change with each vote so that observers cannot establish what people are voting for after they have revealed their first vote. We suggest a (hash) of the consequence field of the poll signed with their private key. This is easily reproducible by a client at a later date with no local storage required.

10\% of the staked tokens are set aside to pay voters when they vote; if a voter has 1\% of the reputation allowed to vote on a decision, they receive 1\% of this pot that is set aside. They receive this payout when they reveal their vote, regardless of the direction they voted in or the eventual result of the decision. This payout regardless of opinion is to avoid us falling victim to the Keynesian beauty contest\cite{KeynesianBeauty}. Any tokens that would have been awarded to users who abstained from voting, or are not revealed in the reveal window, are sent to the root domain funding pot once the poll closes.

Once a vote has been in the reveal phase for 48 hours, a transaction may be made to finalise the vote. Any subsequent reveals of votes do not contribute to the decision being made, but serve only to unlock the user's tokens if it was a token-weighted or hybrid vote (see below). We define quorum to be more than 10\% of the reputation eligible to vote has done so. If quorum is not reached, no changes are made and all participants get their remaining staked tokens returned.\footnote{Some of the staked tokens on the change side will have been used to compensate voters.}

\subsubsection{Consequences of the vote}
If quorum has been reached in a dispute vote, and the `change' side won, then the variable in question is changed, but only if the reputation that voted for this outcome is more than previous votes on the same variable (see Section \ref{sec:repeated-disputes}). If the `keep' side won, then the variable is not changed. In either case, alongside the variable that may or may not have been changed, the fraction of total reputation in the colony that voted for the winning side is noted.

At the conclusion of the poll, losing stakers receive 0-90\% of their staked tokens back and they lose the  complementary percentage of the reputation that was required to stake. The exact amount of tokens they receive back (and therefore reputation they lose) is based on:

\begin{itemize}
 %\item The number of people that voted in a decision
 \item The fraction of the reputation in the colony that voted.
 \item How close the vote ultimately was.
\end{itemize}

At the end of a vote, if the vote was very close, then the losing side receives nearly 90\% of their stake back. If the vote is lopsided enough that the winning side's vote weight ($w$) reaches a landslide threshold ($L$) of the total vote weight, then they receive 0\% of their staked tokens back. $L$ varies based on the fraction of total reputation in the colony that voted ($R$):

\begin{equation}
L = 1 - \frac{R}{3}.
\end{equation}

So for a small vote with little reputation in the colony being allowed to vote, the decision has to be close to unanimous for the losing side to be punished harshly. For a vote of the whole colony, the landslide threshold $L$ reduces to 67\% of the votes --- i.e. the reputation of the colony overall was split 2-to-1 on the decision.

Between these extremes of a landslide loss and a very slim loss, the loss of tokens and reputation suffered by the losing side beyond the 0.1 minimum ($\Delta$) varies linearly:

\begin{equation}
 \Delta = 0.9 \times \min \left\lbrace \frac{w-0.5}{L-0.5}, 1 \right\rbrace
\end{equation}

\noindent and so the total loss ($0.1 + \Delta$) varies between $0.1$ and $1$.

\subsubsection*{What happens to the tokens lost?}
Any tokens lost beyond the initial 10\% are split between the colony and those who staked on the winning side, proportional to the amount they staked. Half of the reputation lost beyond the initial 10\% is given to those who staked on the winning side, and half is destroyed (the colony as a whole having reputation has no meaning, unlike the idea of the colony as a whole owning tokens).

The motivation here is efficiency --- it aims to discourage spurious objections and disputes. A close vote is a sign that the decision was not a simple one and forcing a vote may have been wise. Therefore, the instigators of the dispute should not be harshly punished. On the other hand, if a vote ends in a landslide, it is a sign that the losing side was going up against a general consensus. We encourage communication within the colony. Members should be aware of the opinions of their peers whenever possible before disputes are invoked.

\subsubsection{Repeated disputes}\label{sec:repeated-disputes}
In order to reduce the number of repeated objections and disputes over the same variable, the fraction of total reputation in the colony that voted for the winning side is recorded after every vote. This is the threshold that must be exceeded in any future vote in order to change the variable again. We reiterate that this value is updated after every vote on the variable, even if the decision was to maintain the current value of the variable.

To ensure that the variable can always be changed if necessary, this threshold for changing the variable is ignored if the dispute was raised to the root domain of the colony.

\subsection{Types of vote}
Depending on the context and potential consequences of the vote, Colony supports three types of voting. The type of vote a particular action merits is predetermined based on the action, and is not a choice of the instigator.

\subsubsection{Reputation weighted voting}
Most votes in a colony will be due to disputes related to tasks. In these cases, the weights of the users' votes is proportional to the reputation that each user has in the domain and skill that the vote is taking place in. When such a vote starts, the current reputation state is stored alongside the vote. This allows the current reputation state to be `frozen' for the context of the vote, and prevents unwanted behaviours that might otherwise be encouraged (for example, delaying submission of a task until closer to voting so that the reputation earned has not decayed as much).

When revealing their vote, the user also supplies a Merkle proof of their relevant reputation contained within the reputation state that was saved at the start of the vote. The total vote for the option they demonstrated they voted for is then incremented appropriately.

\subsubsection{Token weighted voting}
Unlike with reputation, we do not have the ability to `freeze' the token distribution when a vote starts. While this is effectively possible with something like the MiniMe token \cite{minime}, we envision token-weighted votes will still be regular enough within a Colony that we do not wish to burden users with the gas costs of deploying a new contract every time.

When conducting a token weighted vote, steps must be taken to ensure that tokens cannot be used to vote multiple times. In the case of `The DAO', once a user had voted their tokens were locked until the vote completed. This introduced peculiar incentives to delay voting until as late as possible to avoid locking tokens unnecessarily.  Our locking scheme avoids such skewed incentives.

Instead, once a vote enters the reveal phase, any user who has voted on that vote will find themselves unable to see tokens sent to them, or be able to send tokens themselves --- their token balance has become locked. To unlock their token balance, users only have to reveal the vote they cast for any polls that have entered the reveal phase --- something they can do at any time. Once their tokens are unlocked, any tokens they have notionally received since their tokens became locked are added to their balance.

It is possible to achieve this locking in constant gas by storing all submitted secrets for votes in a sorted linked list indexed by \ascode{close\_time}. If the first key in this linked list is earlier than \ascode{now} when a user sends or would receive funds, then they find their tokens locked. Revealing a vote causes the key to be deleted (if the user has no other votes submitted for polls that closed at the same time). This will unlock the tokens so long as the next key in the list is a timestamp in the future. A more detailed description of our implementation can be found on the Colony blog \cite{ColonyVoting}.

Insertion into this structure can also be done in constant gas if the client supplies the correct insertion location, which can be checked efficiently on-chain, rather than searching for the correct location to insert new items.

\subsubsection{Hybrid voting}
A hybrid vote would allow both reputation holders and token holders to vote on a decision. We envision such a vote being used when the action being voted on would potentially have a sizeable impact on both reputation holders and token holders. This would include altering the supply of the colony tokens beyond the parameters already agreed (see Section \ref{sec:colony-token-management}) or when deciding whether to execute an arbitrary transaction (see Section \ref{sec:arbitrary-transaction}).

In order for a proposal to successfully pass through a hybrid vote, both the reputation holders and the token holders must have reached quorum, and a majority of both reputation and token holders who vote must agree that the change should be enacted.

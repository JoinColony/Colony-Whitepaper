\section{Overview}

In Section \ref{sec:colonynetwork} we introduce the family of contracts that make up the \textbf{Colony Network}. This includes the \ascode{ColonyNetwork} contract itself, and the \ascode{ColonyFactory} contract which is responsible for creating new instances of the \ascode{Colony} contract. This section also includes a discussion on upgrading these contracts as the network develops, and how we will prevent bugs discovered in the system before it is mature from fatally affecting the network.

Section \ref{sec:clny} introduces the \rcts\ (\textbf{CLNY}) that will underpin the colony network. CLNY is the native token of the \textbf{\rc} --- a colony tasked with maintaining and developing the Colony Network. CLNY will be usable by all colonies, alongside Ether and their own native token, to pay for work. This section also describes how we plan for control of the Colony Network to be transferred to the \rc\ over time.

To begin a collaborative project, a user must send a transaction to the Colony Factory to deploy the contracts for a new colony. These contracts provide the governance mechanisms to enable members to collectively make decisions about the project. This includes decisions about what tasks need to be done, whether tasks should be funded, how much funding they get, and resolving disputes between members.

A Colony is a tool to coordinate effort to achieve a collective goal. We believe that dividing this goal into more achievable units will be essential for the success of any colony; we call these units \textbf{tasks}. Tasks can be assigned as work to members of the colony. The creation, assignment and completion of tasks is the \emph{raison d'{\^e}tre} of a colony. Successful colonies are likely to have many tasks open at any given time; in order to ease the management of tasks, colonies can be divided into \textbf{domains}. These make it easy to group related tasks together and separate them from other unrelated tasks in other domains. Section \ref{sec:colony-structure} describes tasks and domains along with their internal structure and their hierarchy.

Completing a task entitles the member to claim any \textbf{bounty} assigned to the task. Each colony is able to denominate bounties in its own token, Ether, CLNY, and other tokens that adhere to the ERC20 format \cite{erc20} and are whitelisted by the Colony Network. Bounties are assigned to tasks before workers are, with the bounty held in escrow by the colony to ensure the bounty can be claimed when the work is completed. The token allocation system of \textbf{Funding Proposals} is described in section \ref{sec:finance}. Tokens are assigned to domains and tasks on a continuous basis; the funding flows are directed by the members of the colony, and are prioritised by the members' \textbf{reputation}. 

The reputation system is introduced in section \ref{sec:reputation}. Reputation is a key feature of the colony network, and is required to create tasks and domains, as well as to fund them with tokens. Reputation is used to quantify the historical contributions of members to a colony, and to make sure they are justly rewarded. Reputation is not transferable between members, and slowly decays over time. This decay ensures that any reputation a member has is as a result of recent behaviour deemed beneficial to the colony. The calculations involved are too complex to carry out on the Ethereum blockchain. Updates to a member's reputation are calculated off-chain, with an on-chain reporting mechanism secured by economics and game theory. The details of this \textbf{``Reputation Mining''} process are the subject of section \ref{sec:reputationmining}.

Many decisions within a colony are made by consensus. Members are expected to monitor their colleagues actions, but hopefully will only rarely need to intervene. Intervention in this context means `raising an objection' and is the subject of section \ref{sec:objections-and-disputes}. Decision via vote is required infrequently within Colony because it is slow and carries a high coordination cost. A notable exception is in the case of dispute resolution. The \textbf{Dispute Resolution System} allows for any decision to be escalated to a vote of some or all members of the colony. Ballots are weighted meritocratically, according to voters' relevant reputation.

Colonies may be voluntary, non-profit, or for profit. A revenue generative colony may elect to pay out a portion of its revenue to its members. When the colony pays out rewards, the reward a member receives is a function of their combined token and reputation holdings; this ensures those who have contributed the most gain the greatest benefit. Members maximise rewards by contributing to a colony over its whole lifetime rather than simply sitting on accumulated tokens. The details of the \textbf{Reward Payout Process} are contained in section \ref{sec:revenue}.

Throughout this document, various numerical parameters are concretely specified. These values will be subject to empirical review when the colony network begins live operation and any parameter values proposed in this document should be seen as good-faith suggestions, not prescriptions for the final network.

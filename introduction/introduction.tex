\section{Introduction}


\textbf{\emph{Insert some very general waffle-y stuff here containing such feel-good words as `coordination' and `cooperation'}}

\subsection{Overview}

% Project management and collaboration software suites have been around for a long while. What they all lack 
% The backbone of Colony is the \textbf{Colony Network} - a family of smart contracts on the ethereum blockchain that provide the governance mechanisms of the colonies.

We introduce the family of contracts that make up the \textbf{Colony Network} in section \ref{sec:colonynetwork}. This includes the Colony Network contract itself, the Colony Factory contract for creating new colonies as well as the storage and wallet contracts that manage the colonies' data and tokens respectively. This section also includes an outline of how to upgrade the contracts after deployment.

Section \ref{sec:clny} introduces the \textbf{CLNY} token that will be native to the colony network. The CLNY token is the token of the \textbf{\rc} -- a special colony tasked with running the network. Furthermore the CLNY token will be available for all colonies to use alongside Ether and their own tokens. This section also describes how tokens obtained in the \textbf{Colony Crowdsale} can be converted into \rcts\ (CLNY) and how the colony network contracts is to be fully decentralised over time by being passed to the control of the \rc.

To begin collaborating on a project, users would trigger the Colony Factory to form a new colony. The deployed contracts provide the governance mechanisms for the users to make decisions collectively about the project that they are working on together. This includes decisions about what tasks need to be done, whether tasks should be funded and how much funding they get, as well as resolving any disputes between members.

Since the Colony Project is all about coordinating effort to achieve real world results, the `\textbf{Task}' data structure has special significance. Tasks are the individual units of work that are assigned to  users in a colony. Indeed it may be said that the creation, assignment and completion of tasks is the raison d'être of the colony. Successful colonies are likely to have a very large number of tasks open at any one time; in order to ease the management of tasks, colonies can be divided into so-called `\textbf{domains}'. These make it easy to combine tasks under a common heading while simultaneously separating these tasks from other unrelated domains. Section \ref{sec:colony-structure} introduces tasks and domains along with their internal structure and their hierarchy.

Completing work on a task entitles the user to claim a reward or `bounty'. Every colony manages its own token as well as a range of further tokens (Ether, CLNY, ...) that adhere to the ERC20 format \cite{erc20}. In order to have a bounty at all, some of the colony's tokens must have been assigned to the task in the first place. The hierarchical token allocation system of `\textbf{Funding Proposals}' is described in section \ref{sec:finance}. Tokens are assigned to domains and tasks on a continuing basis; the funding flows are directed by the users of the colony and are prioritised by the users' \emph{reputation}. 

The \textbf{Colony Reputation System} is introduced in section \ref{sec:reputation}. Reputation is a prerequisite for creating tasks and domains and directing tokens towards them and is a key feature of the Colony Network. Reputation is used to quantify the historical contributions of users to a colony, and to make sure they are justly rewarded going forward. Reputation is not transferable between users, and slowly decays over time to ensure that any reputation that a user has is as a result of recent behaviour that was deemed beneficial to the colony as a whole. The calculations involved are too complex to carry out on the ethereum blockchain. The Colony Reputation System uses an off-chain calculation and on-chain reporting mechanism secured by economics and game theory. The details of this `\textbf{Reputation Mining}' process are the subject of section \ref{sec:reputationmining}.

Many decisions within a colony are made by consensus. Users are expected to `keep an eye' on what their colleagues are doing, but rarely feel the need to intervene. Intervention in this context means `raising an objection' and is the subject of section \ref{sec:disputes}. Voting as a means of reaching decisions is discouraged in Colony because it requires a lot of effort by many people and is slow and cumbersome; however, voting as a means of conflict resolution is integral to any decentralised colony. The \textbf{Dispute Resolution System} allows for any decision to be escalated to a vote if necessary as the final arbiter. These votes, like so many other decisions in a colony, are weighted by users reputation -- this is a consequence of our desire for decision making to be broadly meritocratic. Thus  reputation is also used when resolving a dispute: relying on the opinions of users who have demonstrated relevant knowledge, weighted appropriately. 

Some colonies may be volunteer efforts or non-profit ventures, while others may indeed be profit making businesses. Colonies that take in revenue will in general also pay out rewards to their token holders. Token balance is not the only determining factor however. When the colony pays out rewards, the more reputation a user has the greater the reward they receive --- i.e. those that have contributed the most gain the greatest benefit. We hope that this incentivises users to keep contributing to colonies over the whole lifetime of the project. The details of the \textbf{Reward Payout Process} are contained in section \ref{sec:revenue}.

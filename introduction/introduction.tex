\section{Introduction}


\textbf{\emph{Insert some very general waffle-y stuff here containing such feel-good words as `coordination' and `cooperation'}}

\subsection{Overview}

In section \ref{sec:colonynetwork} we introduce the family of contracts that make up the \textbf{Colony Network}. This includes the Colony Network contract itself, the Colony Factory contract for creating new colonies, the storage contracts to manage data for a particular colony and the wallet contracts that manage the colony's tokens. This section also includes an outline of how upgrades to the contracts can be applied after the initial deployment.

Section \ref{sec:clny} introduces the \rcts\ (\textbf{CLNY}) that will underpin the colony network. The CLNY token is the token of the \textbf{\rc} --- a colony tasked with maintaining and developing the network. The CLNY token will be able to be used by all colonies, alongside Ether and their own native token, to pay for work. This section also describes how tokens obtained in the \textbf{Colony Crowdsale} can be converted into CLNY, and how we plan for control of the network to pass to the \rc\ over time.

To begin collaborating on a project, a user would send a transaction to the Colony Factory to deploy the contracts for a new colony. These contracts provide the governance mechanisms for users to make decisions collectively about the project. This includes decisions about what tasks need to be done, whether tasks should be funded, how much funding they get, and resolving disputes between members.

A Colony is a tool to coordinate effort to achieve an overall goal. We believe that dividing this goal into more achievable units will be essential for the success of any colony; we call these units \textbf{tasks}. Tasks can be assigned to users of the colony to be completed. The creation, assignment and completion of tasks is arguably the \emph{raison d'être} of a colony. Successful colonies are likely to have many tasks open at any one time; in order to ease the management of tasks, colonies can be divided into \textbf{domains}. These make it easy to group related tasks together and separate them from other unrelated tasks in other domains. Section \ref{sec:colony-structure} describes tasks and domains along with their internal structure and their hierarchy.

Completing a task entitles the user to claim the \textbf{bounty} assigned to the task, if any. Each colony is able to denominate bounties in its own token, in Ether, in CLNY, and other tokens allowed by the network that adhere to the ERC20 format \cite{erc20}. Bounties are assigned to tasks before workers are, with the bounty held in escrow by the colony to ensure the bounty can be claimed when the work is completed. The token allocation system of \textbf{Funding Proposals} is described in section \ref{sec:finance}. Tokens are assigned to domains and tasks on a continuous basis; the funding flows are directed by the users of the colony and are prioritised by the users' \textbf{reputation}. 

The reputation system is introduced in section \ref{sec:reputation}. Reputation is a key feature of the colony network, and is required to crate tasks and domains, as well as to fund them with tokens. Reputation is used to quantify the historical contributions of users to a colony, and to make sure they are justly rewarded. Reputation is not transferable between users, and slowly decays over time. This decay ensures that any reputation a user has is as a result of recent behaviour deemed beneficial to the colony as a whole. The calculations involved are too complex to carry out on the Ethereum blockchain. Updates to a user's reputation are calculated off-chain, with an on-chain reporting mechanism secured by economics and game theory. The details of this \textbf{Reputation Mining} process are the subject of section \ref{sec:reputationmining}.

Many decisions within a colony are made by consensus. Users are expected to monitor what their colleagues are doing, but hopefully will only rarely feel the need to intervene. Intervention in this context means `raising an objection' and is the subject of section \ref{sec:disputes}. Voting as a means of reaching decisions is enforced sparingly within Colony because it requires a lot of effort by many people and is slow. However, voting to resolve conflicts will be required in a colony. The \textbf{Dispute Resolution System} allows for any decision to be escalated to a vote of some or all users in the colony to resolve the disagreement. In order to make decision making close to meritocratic, we weigh ballots in such a vote by the users' relevant reputation.

Some colonies may be volunteer efforts or non-profit ventures, while others may be profit making businesses. Revenue generative colonies may vote to pay out a portion of this revenue to their token holders. When the colony pays out rewards, the reward a user receives is dependent on how many tokens they have and their reputation. The intention here is to ensure those who have contributed the most gain the greatest benefit. We hope that this incentivises users to keep contributing to a colony over the whole lifetime of the project, and not simply sit on accumulated tokens. The details of the \textbf{Reward Payout Process} are contained in section \ref{sec:revenue}.
\subsection*{Introduction}

Colony is the people layer of the decentralised protocol stack.

It allows developers to easily integrate decentralised division of labour, decision making, and financial management into their applications, in a flexible, permissive, and self regulating manner.

Colony brings about a new “Nature of the Firm” \cite{The-Nature-of-the-Firm} by significantly reducing both the transaction costs of the market exchange mechanism for labour, and trust required for people to work together. This innovation makes peer to peer organisations comprising fluid networks of ad hoc contributors possible. Rather than centralised ownership and hierarchical management, smart contracts distribute ownership according to the value each individual contributes, and influence emerges from the bottom up according the merit.

The Colony protocol is intended to be flexible and extensible. This paper seeks to describe and define the full Colony protocol as we currently understand it, but touches only lightly on use cases. As it is the most challenging manifestation to consider, this permutation is designed with low trust, decentralised environments in mind. However, subsets of the functionality in different configurations would be applicable in different circumstances, such as where greater centralisation of trust is desirable.

We envisage this protocol being integrated into a variety of applications. It could form the basis of a decentralised Uber, claims handling in an insurance dApp, or to provide the framework by which a merchant’s guild coordinates in a virtual world.





%\\[3cm] %space before the image

%\begin{center}
%\includegraphics[width=0.7\linewidth]{introduction/Colonylogo.png} 
%\end{center}


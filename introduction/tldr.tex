\subsection*{Introduction}

Ronald Coase argued that entrepreneurs organise production into firms when the transaction cost of coordinating production through the price market mechanism makes the cost of production higher than by employing staff. \cite{The-Nature-of-the-Firm}


However, technology has begun to undermine this model. Companies like Uber and Airbnb are network marketplaces which act as matchmakers between supplier and customer. Whereas before it was impossible for a company to manage relationships with many huge numbers of suppliers without linearly scaling employees to manage them, now software mediates these relationships and the transaction costs of doing so tend to zero.

This new paradigm is however not uniformly beneficial. These platforms are rent seeking gatekeepers 

Colony is the people layer of the decentralised protocol stack. It enables developers to integrate decentralised division of labour, decision making, and collective management of finances into their applications, in a flexible, permissive, and self regulating way.

We envisage this protocol being integrated in a variety of applications. It could form the basis of a decentralised Uber, the way claims are handled in an insurance app, or to provide the framework by which a merchant’s guild coordinates in a virtual world.

The Colony protocol is intended to be flexible and extensible. This paper seeks to describe and define the full Colony protocol as we currently understand it. It explores the full range of functionality designed within the application, but touches only lightly on use cases. As it is the most challenging manifestation to consider, this permutation is designed with low trust, decentralised environments in mind. However, subsets of the functionality in different configurations would be applicable in different circumstances, such as where greater centralisation of trust is desirable.




%\\[3cm] %space before the image

%\begin{center}
%\includegraphics[width=0.7\linewidth]{introduction/Colonylogo.png} 
%\end{center}


\section{Introduction}


\textbf{\emph{Insert some very general waffle-y stuff here}}

\subsection{Colony Network Overview}

Users wishing to collaborate on a project would come together to form a colony, represented by a series of smart contracts on the Ethereum network. These contracts provide the governance mechanisms for the users to make decisions collectively about the project that they are working on together. This includes decisions about what tasks need to be done, whether tasks should be funded and how much they get, and resolving any disputes between members that are encountered.

The key feature of the Colony Network is the underlying reputation system. This is used to quantify the historical contributions of users to a colony, and to make sure they are justly rewarded going forward. It is also used to ensure that, when resolving a dispute, the opinions of users who have demonstrated relevant knowledge are weighted appropriately. Reputation is not transferable between users, and slowly decays over time to ensure that any reputation that a user has is recent and up-to-date.

When the colony pays out rewards, the more reputation a user has the greater the reward they receive --- i.e. those that have contributed the most gain the greatest benefit. We hope that this incentivises users to keep contributing to colonies over the whole lifetime of the project.

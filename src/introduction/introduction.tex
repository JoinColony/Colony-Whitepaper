\section{Overview}

Section \ref{sec:colony-structure} describes the organisational structure of a colony. A colony is a tool to coordinate effort to achieve a collective goal. We believe that dividing this goal into more achievable units will be essential for the success of any colony; we call these units \textbf{tasks}. Tasks can be assigned as work to members of the colony. The creation, assignment and completion of tasks is the \emph{raison d'{\^e}tre} of a colony. Large colonies are likely to have many tasks open at any given time; to simplify the management of tasks (and their budgets), colonies can be divided into \textbf{domains}. These make it easy to group related tasks together and separate unrelated tasks.

Completing a task entitles the member to claim any \textbf{payouts} assigned to the task. Each colony is able to denominate payouts in its own token, Ether, CLNY, and other tokens that adhere to the ERC20 format \cite{erc20} and are whitelisted by the Colony Network. Payouts are assigned to tasks before workers are, with the payout held in escrow by the colony to ensure the payout can be claimed only when the work is completed satisfactorily. The funding allocation system of \textbf{Funding Proposals} is described in Section \ref{sec:funding-queues}. Tokens are assigned to domains and tasks on a continuous basis; the funding flows are directed by the members of the colony, and are prioritised by the members' \textbf{reputation}.

This section also includes a discussion on upgrading these contracts as the network develops, and how we will prevent bugs discovered in the system before it is mature from fatally affecting the network.

The reputation system is introduced in Section \ref{sec:reputation}. Reputation is a key feature of the colony network, and is required to create tasks and domains, as well as to fund them with tokens. Reputation is used to quantify the historical contributions of members to a colony, and to make sure they are justly rewarded. Reputation is not transferable between accounts, and slowly decays over time. This decay ensures that any reputation a member has is as a result of recent behaviour deemed beneficial to the colony. As the calculations involved are too complex to carry out on the Ethereum blockchain, updates to a member's reputation are calculated off-chain, with an on-chain reporting mechanism secured by economics and game theory. The details of this \textbf{Reputation Mining} process are the subject of Section \ref{sec:reputationmining}.

Many decisions within a colony are made by informal consensus. Members are expected to monitor their colleagues actions, but hopefully will only rarely need to intervene. Intervention in this context means `making an appeal' and is the subject of Section \ref{sec:appeals-and-challenges}. Decision via vote is required infrequently within Colony because it is slow and carries a high coordination cost; a notable exception is in the case of appeals resolution. The \textbf{Appeals System} allows for any decision to be escalated to a vote of some or all members of the colony. Ballots are weighted meritocratically, according to voters' contextually-relevant reputation.

Colonies may be voluntary, non-profit, or for profit. A revenue generative colony may elect to pay out a portion of its revenue to its members. When the colony pays out rewards, the reward a member receives is a function of their combined token and reputation holdings; this ensures those who have contributed the most gain the greatest benefit. Members maximise rewards by contributing to a colony over its whole lifetime rather than simply sitting on accumulated tokens. The details of the \textbf{Reward Payout Process} are contained in Section \ref{sec:revenue}.

We of course want people to use Colony for as many different workflows as possible, even those that are not immediately apparent as being able to use Colony. Section \ref{sec:extensions} provides an outline of some more complex behaviours that we have envisaged being possible with the system described here, such as implementing game theoretically-secure task flows and decentralized funding mechanisms.

Section \ref{sec:colonynetwork} outlines the makeup of the \textbf{Colony Network}, introducing the \ascode{ColonyNetwork} contract and the \ascode{Colony} contracts that it spawns. Section \ref{sec:clny} introduces the \rcts\ (\textbf{CLNY}) that will underpin the Colony Network. CLNY is the token of the \textbf{\rc} --- a specially-privileged colony tasked with maintaining and developing the Colony Network. CLNY will be usable by all colonies, alongside Ether and their own token, to pay for work. This section also describes how we plan for control of the Colony Network to be transferred to the \rc\ over time. The generalised equivalent of CLNY for every colony in the network is introduced in Section \ref{sec:colony-tokens}.

Throughout this document various numerical parameters are concretely specified. These values will be subject to empirical review when the Colony Network begins live operation and any parameter values proposed in this document should be seen as good-faith suggestions, not prescriptions for the final network. Similarly, nothing within this document should be understood as a guarantee that any given functionality will be either developed or deployed. In the event that any version of the Colony Network is live on the public Ethereum mainnet, it should be considered as is, and the reader should not be infer, irrespective of the content of this document, that any existing functionality will be upgraded beyond its current state.

\section{Overview}\label{sec:overview}

\subsection{Preamble}

This paper is split into sections corresponding to logically separate components of the Colony Network. The Colony Network’s features however, are in some cases quite interdependent, making it difficult to explain a feature without referencing another before it has been introduced. This overview therefore provides a high level overview of Colony’s capabilities to contextualise the more technical expositions which follow.

Section \ref{sec:overview}, this introduction, lays out the organizational theory which inspires and guides the project. Section \ref{sec:colony-structure}, \textbf{Structure of a Colony}, details the basic functionality of a colony, which can be thought of as an organization’s operating system. Much like a computer operating system provides a secure interface for managing the system’s underlying resources, a colony provides a secure interface for managing an organization’s underlying resources. As with a computer operating system, a colony can be extended with specific applications to implement specific behaviors. We call these applications \textbf{Extensions}, and are the subject of Section \ref{sec:extensions}. Section \ref{sec:colonynetwork} is concerned with the \textbf{Colony Network} as a whole, covering the governance, operation, and token economics of the Colony Network and the special \rc. Section \ref{sec:reputationmining} is specifically concerned with the \textbf{Reputation Mining} process, which powers Colony’s reputation system.

Throughout this document various numerical parameters are specified. These values will be subject to review as we gather evidence from the field, and any parameter values proposed in this document should be seen as good-faith suggestions, not final judgments. Similarly, nothing in this document should be interpreted as a guarantee that any described functionality will be either developed or deployed. The version of the Colony Network that is live on Ethereum mainnet should be considered complete as-is, and the reader should not assume, irrespective of the content of this document, that any existing functionality will be upgraded beyond its current state.

\subsection{Theory of the firm}

Companies exist to coordinate the production of goods and services. Transaction Cost Economics (TCE) theory, popularised by Ronald Coase’s `Theory of the Firm' \cite{The-Nature-of-the-Firm}, postulates that companies form, employ people, and invest in capital because there is a threshold at which it is more efficient to control the factors of production directly than to coordinate production via the market mechanism, once transaction costs are accounted for. These transaction costs come in three flavours:

\begin{itemize}
	\item \textbf{Search \& information}: Costs associated with finding information to inform decisions, and discovering and evaluating suppliers. 
	\item \textbf{Bargaining}: These are costs associated with reaching an agreement with a supplier. Bargaining costs can be very low (e.g. buying a coffee), or very high (e.g. buying a company).
	\item \textbf{Monitoring \& enforcement}: The costs of ensuring adherence to the terms of an agreement (e.g. that widgets are manufactured on time and to the agreed quality). People often deviate from the agreed terms due to chance, negligence, or malice, and potentially high enforcement costs (e.g. legal fees) are required to resolve disputes.
\end{itemize}

TCE theory states that firms are more efficient than the market mechanism at coordinating production due to \textit{imperfect information} and \textit{bounded rationality}. Given perfect information, companies would not be necessary, as market forces would provide the necessary mechanisms to incentivise and coordinate production --- everyone would know the exact value of their and other’s contributions. As this is not the case in traditional markets, these knowledge and trust barriers are overcome by due diligence and contracts, and require a legal system to provide recourse when things go wrong. These processes are expensive, and so traditional firms often find that replacing free-market bargaining with command-and-control hierarchy makes them more efficient and competitive.

As new technologies have improved the diversity and flow of information, new organizations are emerging which are able to merge the efficient decision-making of a market with the shared value-capture of a traditional firm. Gig economy platforms (e.g. Uber, Airbnb), market networks (e.g. eBay, Amazon Marketplace), and cryptocurrencies (e.g. Bitcoin, Ethereum) have demonstrated that if \textit{the product is sufficiently well defined, and the supply sufficiently large, fungible, or diverse}, then it is possible to dramatically reduce the transaction costs of the market mechanism by making search and information discovery easy, bargaining straightforward, and having policing and enforcement provided essentially for free by the platform. This has enabled these new platforms to be orders of magnitude more efficient than had they attempted to coordinate equivalent supply within the hard boundaries of a firm.

\subsection{Confidence and trust}

The firm is able to coordinate complex production at scale by organising labour into a management hierarchy. Seniority within the hierarchy (ideally) represents the amount of confidence the company has in the employee, and in the Platonic ideal of a firm, confidence is a pure function of competence. The more confident the company is in their employee, the higher their competence, and thus the greater their responsibility, influence, and compensation.

Across the internet however, it's hard to have confidence in other people. Up to now we've relied on platform operators to mediate relationships between parties in online transactions (often via various rating and reputation systems), and in some cases (such as payment processing), to underwrite the risk of those transactions. On the blockchain it's even harder, as all you know about a counterparty is that they control a public key. It is difficult to imagine how a traditional organization or hierarchy could exist in this pseudonymous, adversarial environment. A blockchain has no geographical boundaries and cannot differentiate between who or \textit{what} controls public keys. As Richard Gendal Brown put it in his twist on Peter Steiner’s classic meme: \textit{`On the blockchain, nobody knows you're a fridge.'}

Internet Organizations must thus assume the lowest common denominator: that every member is rationally self interested and focussed entirely on maximising personal utility and profit, and given incentives accordingly. This gets to the heart of Colony: a protocol seeking to facilitate the same kind of meritocratic division of labour and authority that the idealised model of the corporate hierarchy should, except from the bottom up, and less prone to error. Decentralised, self-organising companies, where decision-making power derives from a fairly-assessed contribution of value. \\

Work therefore, is where we start. A colony member is compensated for the value they create for the colony, in the form of ETH, any ERC20-compatible token, or \textit{Reputation}, a non-fungible, time-decaying measure of aggregate past contributions.

Active colonies are likely to have a variety of types of work ongoing at any given time; in order to ease the management of work (and their budgets), colonies can be divided into \textit{Domains}. Domains are how you structure your colony. You can think of them as teams, departments, circles, or whatever makes sense in your context. These make it easy to group related tasks together and separate them from other unrelated work in other domains, and make it possible to incorporate contextually-appropriate decision-making logic (in which one domain may be controlled by an administrator, while another is controlled by reputation-weighted voting).

When a colony member gets paid in the colony’s internal token, they also receive Reputation for the \textit{Skills} they used, and in the \textit{Domain} in which the value was created. Reputation is used to quantify the historical contributions of members to a colony, and to make sure they are fairly rewarded. By earning Reputation in a \textit{Skill} (e.g. Javascript) and a \textit{Domain} (e.g. BigCo Client Project), the recipient earns proportional influence in decisions pertaining to those skills and domains. Reputation is not transferable between accounts, and slowly decays over time. This decay ensures that any reputation a member has is as a result of recent behaviour deemed beneficial to the colony (and thus a function of the judgment of the current membership). As the calculations involved are too complex to carry out on the Ethereum blockchain, updates to a member's reputation are calculated off-chain, with an on-chain reporting mechanism secured by economics and game theory (See Section \ref{sec:reputationmining}).

Many decisions within a colony can be made by informal consensus. Members are expected to verify their colleagues conduct, but hopefully will only rarely need to intervene. Intervention in this context means `making a motion' and is the subject of Section \ref{sec:motions-and-disputes}. Decision via vote is infrequent within Colony because it is slow and carries a high coordination cost; costs which are justified in the (hopefully rare) case of dispute resolution. The dispute resolution system allows for many kinds of decisions to be put to a contextually-relevant vote of some or all members of the colony. Ballots are weighted meritocratically, according to voters’ contextually-relevant reputation.

Colonies may be voluntary, non-profit, or for profit. A revenue-generating colony may elect to pay out a portion of its revenue to its members. When the colony pays out rewards, the amount a member receives is a function of their combined token \textit{and} reputation holdings; this ensures those who have contributed the most gain the greatest benefit. Members maximise rewards by contributing to a colony over its whole lifetime (thus maintaining high levels of reputation) rather than sitting on early accumulations of tokens. The details of the rewards payout process can be found in Section \ref{sec:revenue}.

We want people to use Colony for as many different workflows as possible, even those that are not immediately apparent as being able to leverage the Colony protocol. Section \ref{sec:extensions-misc} provides a brief outline of some more complex behaviours that we have envisaged as possible with the tools described here.
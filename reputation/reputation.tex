\section{Reputation}
\subsection{What is Reputation?}
blabla text here ... score associated to an account... cannot be transferred... used as voting weight... hoepfully leading to meritocratic decision making.
FIXME FIXME

There are many reputation scores per user ... local to domains or skills.... not one global score. Also Reputation decays with time.



Reputation earned in Colony will primarily be earned from completing tasks, and the amount of reputation earned will be based on a combination of the performance of the user completing the task, and the payout associated with the task itself.

We note this creates a problem we have affectionately taken to calling the ‘Palo Alto Problem’, whereby users who are paid more for the same work - such as those living in Palo Alto may be compared to those in South America - earn more influence in the colony. While we would like to solve this problem, we have been unable to do so to date; any possible solution either seems to introduce a great deal of complexity to the system or provide the ability for a bad actor to accrue undue influence in the colony.
\subsubsection{Reputation per Domain}
Users have reputation local to any domain. Earning rep in a domain earns you rep in any parent domain. Losing rep also affects all children.
Top Level reputation ...

beyond this we also want skill tags:

\subsubsection{Skill hierarchy}

The existence of domains that form the organisational hierarchy of a colony was described in section \ref{sec:domains}. In addition to this organisational hierarchy (which is unique to each colony), the Colony Network also maintains a hierarchy of skill domains (or just `skills'), which is available for all colonies to use. A user's good or bad activity associated with either of these hierarchies will earn or lose them the corresponding reputation, though only in a single colony.\footnote{It is hopefully clear that allowing users to earn reputation in one colony and then use it to influence decisions in another would be ripe for abuse.}

For example, when a task is created, as well as being placed in a particular domain in the colony, it is also tagged with a skill from the skill hierarchy. When the worker earns reputation for succesfully completing the task, they will earn reputation in all the relevant domains and skills. Conversely, if they are to lose reputation because their work is found inadequate, they will do so in all the relevant domains ad skills.

\subsection{Earning reputation}
The only way that reputation is created in a colony is when it is FIXME

\subsubsection{Bootstrapping reputation}

Since a large portion of a colony's decision making procedure rests on reputation weighted voting, we are presented with a bootstrapping problem for new colonies.
When a colony is new, no-one has yet completed any work in it and so nobody will have earned any reputation. Consequently,no dospites could be resolved as no-one would be able to vote on them. \\
Therefore, when a colony is created, the creator can nominate addresses to have initial reputation assigned to them to allow the colony to bootstrap itself. There will be a global limit on the reputation that can be assigned in this manner in order to prevent an extreme reputation aristocracy. Given that reputation decays over time, this initial bootstrapping of reputation will not have an impact on the long-term operation of the colony. \textbf{This is the only time that reputation can be created without associated work being done.}

At first glance it may appear as if the same bootstrapping problem presents itself whenever a domain is created - if the domain has had no work done in it, then who has the authority to make decisions? We do not wish to allow the creation new reputation here, as this would devalue reputation already earned in the colony by users completing work. Luckily we can proceed without any new reputation: we simply accept the fact the new domain has no reputation in it. The colony is still able to make decisions and resolve disputes, because any objections can be escalated to a parent domain if necessary. Furthermore, even this escalation is not necessarily required in the event of a disagreement, because, even if there is no reputation in the domain to contribute to the decision, users will still be able to vote based on their reputation in relevant skills.

\subsubsection{Earning Reputation by contributing to a task}
admin, worker, evaluator.... FIXME

\subsubsection{Earning Reputation as a result of Disputes}
see title.



\subsection{Losing Reputation}

\subsubsection{Reputation Decay}
All reputation decays over time.\\
Every 600000 blocks, a user’s reputation in any domain or skill decays by a factor of 2. This decay occurs continuously, rather than being a step change every three months to ensure there are as few incentives to earn reputation at any particular time.

\subsubsection{Losing Reputation due to a Negative Evaluation}
working on a task. bad review. lose rep....


\subsubsection{Losing Reputation due to Bad Behaviour}
disputes and how they affect rep....


Whenever a user stakes tokens, they are also risking their reputation. If the tokens are lost, then they also lose reputation. Reputation is not staked, and so while at risk it can still be used by users to vote on decisions where that reputation is relevant - they only lose the voting power associated with the reputation once it is lost. However, it should be noted that the same is not true of colony tokens - once staked, their voting power is lost. However, we expect that users will only ever stake a small proportion of their tokens - only small amounts of tokens are staked in the Colony Network to ensure a tangible financial punishment to the staking user can be made if they are a badly behaving actor.

The amount of tokens to be staked and reputation that can be lost depend on the context of the action being taken. The greater the proportion of reputation in the colony that is potentially inconvenienced by the action being taken in the event that it is fraudulent, the greater the reputation and tokens that must be staked by the user to take the action.








\subsection{Reputation mining}\label{sec:reputationmining}


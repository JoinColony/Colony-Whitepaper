\section{The Reputation System}\label{sec:reputation}
\subsection{What is Reputation?}\label{subsec:what-is-reputation}

Reputation is a number associated with an account which attempts to quantify the merit of a user's recent contributions to a colony. Reputation is used to weight a user's influence in decisions related to the expertise they have demonstrated, and to determine amounts owed to a colony's members when rewards are disbursed (see Section \ref{sec:claimrewards}). Because reputation is awarded to users by either direct or indirect peer approval of their actions, influence and rewards are distributed according to merit.

Colony aims to be broadly meritocratic. Consequently, the majority of decisions in a colony are weighted by the relevant reputation.

Unlike tokens, reputation cannot be transferred between addresses %\footnote{If an address holding reputation is a contract, rather than an externally-owned account, then \emph{control} of that reputation may be passed between parties.}
; it represents an appraisal of the address's activities by their peers. Reputation must therefore be earned by direct action within the colony. Reputation that is earned will eventually be lost through inactivity, error, or malfeasance; a description of how reputation is gained and lost is given in Section \ref{sec:earning-losing-rep}.

\subsubsection{Reputation by Domain}\label{sec:rep-by-domain}
The hierarchical structure of a colony was described in Section \ref{sec:domains}. Reputation is earned in this hierarchy, and a user has a reputation in all domains that exist --- even if that reputation is zero. When a user earns or loses reputation in a domain, the reputation in all parent domains changes by the same amount. In the case of a user losing reputation, they also lose reputation in all child domains. When reputation is lost in a domain with children, all child domains lose the same fraction of reputation. If a reputation update would result in a user's reputation being less than zero, their reputation is set to zero instead.

An example makes this clearer. Suppose a colony has a `development' domain which contains a `backend' domain and a `frontend' domain, as in Figure \ref{fig:domainhierarchysample} on page \pageref{fig:domainhierarchysample}. Any time a member of the colony earns reputation for work completed in the backend domain, it will increase their backend reputation, their development reputation and their reputation in the all-encompassing top-level domain of the colony. Reputation earned in the development domain will only increase the development and top-level domain reputation scores of the user.

Later, the user behaves badly in the `development' domain, and they lose 100 reputation out of the 2000 they have in that domain. They also lose 100 reputation in the parent domains, and 5\% $\left(\frac{100}{2000}\right)$ of their reputation in each of the child domains of the `development' domain (which in this example, includes all of the Frontend, Backend, Node.js and Ruby domains). 

\subsubsection{Reputation by Skill}\label{sec:rep-by-skill}

We envision domains to mostly be used as an organisational hierarchy within a colony. However, this would not necessarily capture the \emph{type} of work that a user completed to earn their reputation. If the domain were a project, with tasks corresponding to both design and development work, reputation earned by completing tasks related to these skills would not be distinguishable. To have a more granular account of the work a user completes to earn their reputation, a skill hierarchy is also maintained.

This global hierarchy of skills is available to all colonies, and is curated and maintained by the \rc. When a task is created, as well as being placed in a particular domain in the colony, it is also tagged with a skill from the skill hierarchy. When the worker earns reputation for successfully completing the task, they will earn reputation in the skill the task was tagged with and all parent skills. This is in addition to the reputation earned in the relevant domains. Conversely, if they are to lose reputation because their work is found inadequate, they will lose a proportional amount reputation from all child skills of the tag (if any exist), as is the case with the domain reputation. There is a top-level skill analogous to the top-level domain in a colony, of which all skills are descendants.

Even though the skill hierarchy is universal, reputation earned in the skill hierarchy is unique to each colony. Earning reputation in a skill in one colony has no effect on the user's reputation in that skill in any other colonies.

\subsubsection{Reputation by Colony}\label{sec:rep-by-colony}
A user's total reputation in a colony is the sum of their reputation in the top-level skill and the top-level domain. This is the reputation they will be voting with in any decisions that require input from everyone in the colony. Reputation in a colony has no effect outside the colony. In particular, reputations held in one colony have no bearing on reputations held by the same account in another colony.

\subsection{Earning and losing reputation}\label{sec:earning-losing-rep}
There are three ways to earn reputation in a colony.\footnote{The \rc\ is a special case in which reputation may also be earned by Reputation Mining --- see Section \ref{sec:reputationmining}.} The first is being involved with a successfully completed task. The second is through the dispute process. In both of these cases, the user has contributed usefully to the colony and is rewarded accordingly. The third way to earn reputation is upon the creation of a colony and the associated bootstrapping process (see Section \ref{sec:bootstrapping-rep}).

Reputation losses can arise from a user being found responsible for a badly executed task, or being involved in the dispute process and the dispute being resolved against them. In addition, all reputation earned by users is exposed to a continual decay over time. 

The rest of this section outlines each of these mechanisms, with references to the more detailed descriptions given elsewhere where appropriate.

\subsubsection{Reputation change from contributing to a task}\label{sec:earning-rep-from-task}
Each task requires three roles to be assigned: the manager, the worker and the evaluator (as described in Section \ref{sec:tasks}). If the bounty for the task is denominated in the colony's token, each of these roles are eligible to earn reputation when the task is completed as long as their work was well received.

The performance of the user who has completed the work is established when the work is submitted and then evaluated. At this point, the evaluator grades the work submitted by the worker, and the worker rates the manager's ability to coordinate delivery of the task\footnote{These scores should be submitted using a pre-commit and reveal scheme to ensure secrecy during the rating process and avoid retaliatory grading in the event that the manager and evaluator are the same person, which we expect to be a reasonably common occurrence. In the event of a user not committing or revealing within a reasonable time, their rating of their counterpart is assumed to be the highest possible and they receive a mild reputation penalty. } on a scale of one to three points.

In the case of the evaluator, a rating of 1 point counts as them rejecting the work and a score of 2-3 points counts as accepting the work. Therefore ratings and the reputation lost or gained follow these rules:
\begin{itemize}
 \item[]1 point:\phantom{s} User was unable to complete the task. Reputation penalty equal to the token payout.
 \item[]2 points: User completed the task acceptably. Reputation gain equal to the token payout.
 \item[]3 points: User completed the task to a higher standard than requested. Reputation gain equal to $ 1.5 \times \text{token payout} $.
\end{itemize}

The manager receives reputation following the same rules as the worker, however they only earn this reputation in the current (and all parent) domains, not in the skill reputation hierarchy as they have not actually done the task. While it is likely some knowledge is required to coordinate delivery of the task, this is not always the case; we believe that skill reputation should exclusively demonstrate ability to perform tasks.

Upon completion of a task, the evaluator also earns reputation based on their token reward. There is no explicit rating of the evaluator, but as with all other payments and rewards, an objection can be raised before a payout occurs; if the evaluation is changed by such an objection, the evaluator's reward is reduced or turns in to a loss of reputation. For all participants, reputation updates occur and payouts are made available only \emph{after} the objection window (described in Section \ref{sec:tasks}) has closed and all disputes  (described in Section \ref{sec:objections-and-disputes}) have been resolved at the end of the task. The reputation updates and payouts are based on the final state of the task and the difference, if any, between the initial gradings and final state of the task.

\subsubsection{Reputation change as a result of Disputes}\label{sec:earning-rep-in-disputes}
If a dispute occurs, causing a vote among some portion of the colony, each side will have had to stake some number of tokens. Those who staked on the side determined to be right gain their stake back, plus some tokens that have been lost by the losing side. There will also be a reputation change as a result --- those on the losing side will lose some reputation, and some of that will be gained by the winning side. Section \ref{sec:objections-and-disputes} provides a full description of the dispute mechanism and the amount of tokens and reputation each side loses and gains.

We note here that, unlike tokens, reputation is never staked during this process. A user must hold greater than a minimum threshold of reputation in order to be eligible to stake tokens and take certain actions, and this threshold only applied at the time the action is taken. If their stake is lost, they will lose some reputation as a consequence for being on the losing side. It is possible that the user will not have enough reputation to lose by the time this reputation update occurs due to bad behaviour elsewhere. This is acceptable from a game-theoretic perspective; the user has still had to stake and lose tokens, and these cannot be `double-spent'.

\subsubsection{Bootstrapping reputation}\label{sec:bootstrapping-rep}
Since a colony's decision making procedure rests on reputation weighted voting, we are presented with a bootstrapping problem for new colonies. When a colony is new, no-one has yet completed any work in it and so nobody will have earned any reputation. Consequently, no objections can be raised and no disputes can be resolved as no-one is able to vote. Then, once the first task is successfully completed, that user has a dictatorship over decisions in the same domains or skills until another user earns similar types of reputation.

To prevent this, when a colony is created, the creator can choose addresses to have initial reputation assigned to them in the colony-wide domain to allow the colony to bootstrap itself. The reputation assigned to each user will be equal to the number of tokens received, i.e. if a member receives ten tokens, they also receive ten reputation in the colony-wide domain. Given that reputation decays over time, this initial bootstrapping will not have an impact on the long-term operation of the colony. This is the only time that reputation can be created without an associated task being paid out. Users receiving reputation are presumably the colony creator and their colleagues, and this starting reputation should be seen as a representation of the existing trust which exists within the team. 

We note that the same is not required when a new domain is created in a colony. We do not wish to allow the creation of new reputation here, as this would devalue reputation already earned elsewhere in the colony. This bootstrapping issue is resolved by instead using reputation within the parent domain, when a child domain contains less than 10\% of the reputation of its parent domain. A domain below this threshold cannot have domains created under it.

\subsubsection{Reputation Decay}
All reputation decays over time. Every 600000 blocks, a user's reputation in every domain or skill decays by a factor of 2. This decay occurs every hour, rather than being a step change every three months to ensure there are minimal incentives to earn reputation at any particular time. This frequent, network-wide update is the primary reason for the existence of the reputation mining protocol, which allows this near-continuous decay to be calculated off-chain without gas limits, and then realised on-chain. 

The decay serves multiple purposes. It ensures that reputation scores represent \emph{recent} contributions to a colony incentivising members to continually contribute to the colony. It further ensures that wild appreciations in token value (and the corresponding decrease in tokens paid per task) do not permanently distort the distribution of reputation but instead serves to smooth out the effects of such fluctuations over time.

\subsection{On-chain representation of skills and domains}\label{subsec:on-chain-representation-of-skills}
In the context of reputation, domains and skills are the same, differing only in that domains are colony-specific categorisation and skills are universal categorisation. In this subsection, each instance of `skill' should be taken to mean `skill or domain'.

Each skill that reputation can be earned in is assigned a \ascode{rep\_id} that is unique across the whole network. When a skill is created, additional properties are recorded and initialised.
\begin{equation*}
  \ascode{skill\_id} \rightarrow 
  \begin{cases}
    \ascode{n\_parents} &	\textnormal{total number of parents}\\
    \ascode{parent\_n\_id} &	\parbox[t]{.6\linewidth}{\textnormal{the \ascode{rep\_id} of the $n^{\rm th}$ parent, where $n$ is an integer power of two larger than or equal to 1. }}\\
    \ascode{children}\left[\cdots\right] &	\textnormal{array of \ascode{rep\_id}s of all child skills}\\
    \ascode{n\_children} &	\textnormal{total number of child skills}
  \end{cases}
\end{equation*}
Upon creation, \ascode{children[]} and \ascode{n\_children} are empty\watermark. These two fields in all parents are updated with the \ascode{skill\_id} of the new skill on creation.\footnote{We acknowledge that this is fundamentally gas limited, but the only consequence of this will be the inability to create new skills once the maximum depth allowed by the block size is reached. Back-of-the-envelope calculations suggest this corresponds to a depth of around 80, which we don't believe our users will be limited by.}

Storing these pieces of data on-chain is required, as they are used by the reputation mining protocol (see next section) and the procedures for escalating disputes (see Section \ref{sec:objections-and-disputes}). They are stored under the control of the \ascode{ColonyNetwork} contract.

\subsection{Reputation update log}\label{subsec:reputation-update-log}

Whenever an event that causes one or more users to have their reputation updated in a colony occurs, a corresponding entry is recorded in a log in the \code{ColonyNetwork} contract. Each entry in the log contains

\begin{itemize}
\item The user suffering the reputation loss or gain.
\item The amount of reputation to be lost or gained.
\item The type of reputation to be lost or gained.
\item The colony the update has occurred in.
\item How many reputation entries will need to be updated (including parent, child and colony-wide total reputations). This is the motivation for storing \ascode{n\_parents} and \ascode{n\_children} for each skill and domain, as described in Section \ref{subsec:on-chain-representation-of-skills}.
\item How many total updates to reputations have occurred before this one in this cycle, including decays and updates to parents and children.
\end{itemize}

If the reputation update is the result of a dispute being resolved (as outlined in Section \ref{sec:earning-rep-in-disputes}), then instead of these first three properties, there is a reference to the dispute-specific record of stakes in the relevant colony. For the structure of this log, and an explanation of the way that it allows individual updates to be extracted in constant gas, see Appendix \ref{appendix:rep-transfer}.

This log exists to define an ordering of all reputation updates in a reputation update cycle that is accessible on-chain. In the event of a dispute during the reputation mining protocol (described in Section \ref{sec:reputationmining}), the \ascode{ColonyNetwork} contract can use this record to establish whether an update has been included correctly.

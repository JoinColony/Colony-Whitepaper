\section{Structure and Hierarchy within a Colony}\label{sec:colony-structure}
Colonies exist to enable collaboration between their members, and direct collective efforts towards some common goal(s). Facilitating effective division of labour is therefore, one of the most important capabilities of the Colony protocol.

Work is divided into discrete `tasks', which can then be organised into `domains'.

\subsection{Tasks}\label{sec:tasks}

The smallest structural unit in a colony is the `task'. A task represents a unit of work requiring no further subdivision or delegation. A task has three roles associated with it:
\begin{itemize}
\item An manager --- someone responsible for defining, and coordinating the delivery of the task.
\item A worker --- someone responsible for executing the task.
\item An evaluator --- someone responsible for assessing whether the work has been completed satisfactorily.
\end{itemize}

These roles are assigned to addresses. While it is anticipated that roles within tasks will likely be assigned to individuals, there is nothing to prevent these addresses being contracts under the control of multiple people.\footnote{With the protocol described in this version of the document, any reputation earned would be assigned to the contract in question and not able to be moved to the appropriate users. We would expect some further developed version of the Colony Network to be able provide this functionality to users.}

The manager (usually the creator of the task) is responsible for selecting the evaluator and worker and setting additional metadata for the task:

\begin{itemize}
\item A due date, represented by a block number.
\item Bounties for each of the manager, the worker and the evaluator.
\item A specification, or brief: to be used by the worker to guide the work, and by the evaluator for assessing the satisfactory completion of the task.
\end{itemize}

In order to create a task, the manager must stake some number of Colony Tokens. This (small) stake is used to discourage spam, and provide a mechanism whereby the manager can be punished for bad behaviour. 

Defining what the payouts for each role should be, of course, does not provide the funds --- this must be done through the funding mechanisms in Colony (see Section \ref{sec:finance}). A funding proposal can be made by anyone, but the manager is the natural person to do so.

Assigning the worker can only occur after the funds to satisfy the proposed payout have been received by the task, and both the manager and proposed worker have indicated they are happy with the assignment. After that point, the terms of the assignment may be altered either by mutual consent, or via the dispute process.

After the task has been assigned, the worker may make a `final submission', which includes some evidence that the work has been completed.

Once the due date has passed or the worker has made their submission, the evaluator may rate the work. Regardless of whether the rating is positive or not, the task enters a state in which \textbf{objections} to the final state of the task can be made and \textbf{disputes} can be initiated (see Section \ref{sec:objections-and-disputes}). Once three days have elapsed, no more objections or disputes can be raised. Once all pending disputes related to the task are resolved, the parties involved get punished or rewarded based on the final state of the task.

If the evaluator's rating for the work is changed as a result of a dispute, they get a reduced payout based on the discrepancy between their original score and the score that their peers determined to be more appropriate. If an objection has determined the manager or evaluator should be punished, they lose their stake, otherwise the stake is able to be reclaimed. The worker is paid based on the final score awarded to their work, taking into account the result of any disputes.

\subsection{Domains}\label{sec:domains}

Of course, without structure, a large colony could quickly become difficult to navigate due to the sheer number of tasks —-- domains solve this problem. A domain is sort of like a folder in a shared filesystem, except instead of containing files and folders, it can contain sub-domains and tasks. This simple modularity enables great flexibility as to how an organisation may be structured. A toy example is shown in in Figure \ref{fig:domainhierarchysample}. 

\begin{figure}[h]
    \centering
 \begin{tikzpicture}
   \node[ellipse,draw, dashed] at (0,0) (tld) {Colony Domain};
   \node[ellipse,draw, dashed] at (-2,-2) (design) {Design};
   \node[ellipse,draw, dashed] at (2,-2) (development) {Development};
   \node[ellipse,draw, dashed] at (0.5,-4) (frontend) {Frontend};
   \node[ellipse,draw, dashed] at (4,-4) (backend) {Backend};
   \node[ellipse,draw, dashed] at (3,-6) (nodejs) {Node.js};
   \node[ellipse,draw, dashed] at (5,-6) (ruby) {Ruby}
    (tld.-120) edge[->, bend left=45, in=-120] (design.north)
    (tld.-60) edge[->, bend left=45, out=-60, in=120] (development.north)
    (development.-120) edge[->, bend left=45, in=-120] (frontend.north)
    (development.-60)  edge[->, bend left=45, out=-60, in=120] (backend.north)https://www.sharelatex.com/project/58d0130fd55fbc07070d7c33
    (backend.-120) edge[->, bend left=45, in=-120] (nodejs.north)
    (backend.-60)  edge[->, bend left=45, out=-60, in=120] (ruby.north);
 \end{tikzpicture}
 \caption{Parts of a possible domain hierarchy for a colony developing a web service.}
 \label{fig:domainhierarchysample}

\end{figure}

This compartmentalisation of activity provides a tangible benefit to the colony as a whole. When objections are raised, they can be raised to a specific level in the colony's domain hierarchy. This means that people with relevant contextual knowledge can be targeted for their opinion, and that when a dispute occurs, the whole colony is not required to vote in the dispute. Rather, only members with the relevant experience are asked for their opinion.

It is ultimately up to individual colonies to decide how they wish to use domains --- some might only use them for coarse categorisations, whereas others may use them to precisely group only the most similar tasks together, or even multiple tasks that other colonies would consider a single task. We aim to provide a general framework that colonies may use however they see fit, and to only be prescriptive where necessary.

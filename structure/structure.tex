\section{Structure and Hierarchy}

\subsection{Tasks}\label{sec:tasks}

The fundamental unit of structure in a colony is the `task'. A task has three roles associated with it:
\begin{itemize}
\item An administrator - someone responsible for defining the task
\item A worker - someone responsible for executing the task
\item An evaluator - someone responsible for checking the work has been completed
\end{itemize}

These roles are assigned to addresses. While we anticipate that tasks will be mainly be defined such that each of these roles is assigned to an individual in the first instance, there is nothing preventing these addresses being contracts under the control of multiple people.\footnote{With the protocol described in this version of the document, any reputation  earned would be assigned to the contract in question and not able to be moved to the appropriate users. We would expect some further developed version of the Colony Network to be able provide this functionality to users.}

The administrator (the creator of the task) is responsible for selecting the evaluator and worker (by whatever means they deem appropriate) and setting additional metadata for the task:

\begin{itemize}
\item A due date (represented by a block number)
\item Payouts for each of the administrator, the worker and the evaluator. We anticipate that the worker should be the role to get the bulk of each payout, as they are actually doing the work. 
\item A specification or brief for the task, to be used by the worker to guide the work, and by the evaluator for deciding if the work has been completed.
\end{itemize}

In order to create a task, the administrator must stake 0.01\% of the Colony Tokens in existence. This small stake is used to help discourage spamming of nonsense tasks, and provide a mechanism whereby the administrator can be punished upon bad behaviour. 

Defining what the payouts for each role should be, of course, does not provide the funds - this must be done through the funding mechanisms in Colony (see section \ref{sec:finance}). While the administrator is not the only person who is able to make the funding proposal - which can be made by anyone - they are the natural person to do so.

Formally defining the worker is irreversible, and can only occur after the funds to satisfy the proposed payout has been received by the task, and both the administrator and proposed worker have accepted they are happy with the assignment. After that point, the worker and payout cannot be changed, though other variables associated with the task can be.

Once all these elements have been set and funding is secured, the worker is able to --- at any point --- make a ‘final submission’, which includes some proof that the work has been completed. This is a generic field that can be use for any proof that the work has been completed, but most likely should be an IPFS or Swarm hash referring to a longer file rather than a large proof that would be expensive to store on-chain.

Once the due date has passed or the worker has made their submission, the evaluator is able to grade the work. Regardless of whether the grading is positive or not, the task then enters a state where objections or `disputes' (see section \ref{sec:disputes}) can be made over the final state of the task but no other changes can be made. Once three days have passed, no more disputes can be raised. Once all pending disputes related to the task are resolved, the parties involved get punished or rewarded based on the final state of the task.

If the evaluator’s grading for the work is changed as a result of a dispute, they get a reduced payout based on the discrepancy between their original score and the score that their peers determined to be more appropriate. If a dispute has determined the administrator should be punished, they lose their stake, otherwise the stake is returned. The worker is paid based on the final score awarded to their work, taking into account the result of any disputes.

\subsection{Domains}\label{sec:domains}

Of course, in a large colony, it would quickly become difficult for users to find relevant tasks just because of the sheer number of tasks. We therefore encourage users to create domains in their colony. These domains can contain both sub-domains and tasks. 

This compartmentalisation of activity in a colony provides a tangible benefit to the colony as a whole. When objections are raised, they can be raised to a specific level in the structural hierarchy within a colony. This means that people with relevant contextual knowledge can be targeted for their opinion, and also means that when a dispute occurs the whole colony does not have to vote in the dispute. Only those users with the relevant skills are asked for their opinion.

We also note that on some level, it is up to individual colonies to decide how they wish to use domains - some might only use them for coarse categorisations, whereas others may use them to very precisely group only the most similar tasks together, or even multiple tasks that other colonies would consider a single task. We aim to provide a general framework that colonies can use how they see fit, and to only be prescriptive in the service’s use where necessary.



\section{Gas-efficient reputation penalty in dispute resolution}

Once a dispute has been raised and settled one way or the other, the users on the losing side will lose reputation if they have it to lose. If those users lose reputation, some of those on the winning side will therefore gain reputation. If there is a dispute in the reputation mining mechanism, we must be able to calculate on-chain, in a gas-efficient way, whether a specific user was able to lose reputation or not at that time. All of the reputation losses that each user who was involved in the dispute is only represented by a single entry in the `reputation update log`, and so it is necessary to expand upon the process used to do this.

Given that users are able stake small amounts on each objection, an arbitrarily large number of users could theoretically be involved. We must therefore not have any (on-chain) loops in this implementation.

\subsection{Staking}

Let's consider a situation where three users initiate an objection, and two users oppose it with matching funds. In the dispute, the initiating users were found to have been wrong and so will lose their stake.

\begin{table}[h]
\centering
\caption{}
\begin{tabular}{|c|c|c|c|c|c|}
\hline
Stake \# & User  & Staked Amount & $\Sigma^+$ & $\Sigma^-$ \\ \hline
1 & A & 100           & 100                      & 0                                                                       \\ \hline
2 & B & 200           & 300                      & 0                                                                       \\ \hline
3 & C & 300           & 600                      & 0                                                                       \\ \hline
4 & D & -150          & 600                      & -150                                                                    \\ \hline
5 & E & -450          & 600                      & -600                                                                    \\ \hline
\end{tabular}
\end{table}

Let's also assume that all users have the appropriate amount reputation to lose. There are four transfers of reputation that must occur here:

\begin{enumerate}
\item User A loses 100 reputation to User D
\item User B loses 50 reputation to User D
\item User B loses 150 reputation to User E
\item User C loses 300 reputation to User E
\end{enumerate}

Indeed, in a group of $m$ people where some owe the others a debt, the maximum possible of transfers required to make everyone whole is equal to $m-1$. If the reputation being lost has $p$ parents and $c$ children, then there are $2 + 2p + c$ reputation updates that must occur at each of these steps. There are therefore $\left(m-1\right)\times\left(2+2p+c\right)$ reputation updates in total. In the event of a disagreement regarding the reputation state, we must be able to access the $n$th update directly when calculating an update on-chain. This is made possible by additional logging of data when stakes are made.

When a user stakes and opposes some existing stake that does not yet have a counterparty, we record the stakes that it is matching against as well as any remainder.

\begin{table}[ht]
\centering
\caption{}
\begin{tabular}{|c|c|c|c|c|c|}
\hline
Stake & Match From & Match To & Remainder & Tx \# From & Tx \# To\\ \hline
-150  & 1          & 2        & 50      & 1 & 2 \\ \hline
-450  & 2          & 3        & 0       &  3 & 4 \\ \hline
0\tablefootnote{The justification for this line being present is given in section \ref{sec:exactMatching}}  & 0         & 0        & 0 & 5 & 5            \\ \hline
\end{tabular}
\end{table}

When staking, the user supplies the 'Match From' and 'Match To' arguments. These can be checked to be correct on-chain in constant gas by using the values of $\Sigma^+$ and $\Sigma^-$ recorded alongside previous stakes. Then, when a \rcth is asked to prove a particular transaction has been included, they can point to the row in this log that contains that transaction without the contract having to iterate over an arbitrarily long list.

\subsection{Exact matching}\label{sec:exactMatching}

For the `reputation update log' to work correctly, we must know exactly how many reputation updates we have to consider. In the above example, it was $4\times (2+2p+c)$, which could be calculated and recorded easily in the update log. However, consider an example where the staked amounts were

\begin{table}[ht]
\centering
\caption{}
\begin{tabular}{|c|c|c|c|c|c|}
\hline
Stake \# & User  & Staked Amount & $\Sigma^+$ & $\Sigma^-$ \\ \hline
1 & A & 100           & 100                      & 0                                                                       \\ \hline
2 & B & 200           & 300                      & 0                                                                       \\ \hline
3 & C & -100           & 300                      & -100                                                                       \\ \hline
4 & D & -200          & 300                      & -300                                                                    \\ \hline
\end{tabular}
\end{table}

Even though there are four people, only two transfers are required --- from user A to user C, and from user B to user D. This is because the users have accidentally matched themselves exactly, and so one transaction makes two users `whole'. In order to accommodate this possibility in the reputation update log, we insert dummy reputation transfers in the log whenever an exact match occurs:

\begin{table}[ht]
\centering
\caption{}
\begin{tabular}{|c|c|c|c|c|c|}
\hline
Stake & Match From & Match To & Remainder & Tx \# From & Tx \# To \\ \hline
-100  & 1          & 1        & 0   & 1 & 1     \\ \hline
 0 & 0          & 0        & 0      & 2 & 2   \\ \hline
-200 & 2          & 2        & 0    & 3 & 3    \\ \hline
 0 & 0          & 0        & 0      & 4 & 4   \\ \hline
\end{tabular}
\end{table}

These dummy insertions occur whenever the remainder is 0 --- i.e. the stake has matched  the first unmatched stake exactly (or its remainder). This ensures that this log always describes as many transactions as there are people (the last entry is always a dummy transaction as the final transaction will always make two users whole). This means that regardless of how the users have matched up against each other, the event that is recorded in the reputation update log will have a known number of transactions equal to the number of staking users, even if some of those are `null' transactions.

\subsection{Generalisation}

This procedure can also be extended to account for stakes being made in aribtrary orders (i.e. not all of one side followed by all of the other). This can be achieved by making the indices in the `Match From' and `Match To' columns only refer to stakes on one side or the other, and keeping matching lists that map those indices onto the stakes. Altering our first example slightly, we might end up with the stakes in table \ref{appa:modifiedexample} and the log entries in table \ref{appa:modifiedexamplelog}. The list of positive stakes would be [1, 3, 4] and the list of negative stakes would be [2, 5]. The `Match From' and `Match To' indices then refer to the list corresponding to the opposite sign of the stake currently being considered.

\begin{table}[ht]
\centering
\caption{}
\label{appa:modifiedexample}
\begin{tabular}{|c|c|c|c|c|}
\hline
User  & Staked Amount & $\Sigma^+$ & $\Sigma^-$ \\ \hline
A & 100           & 100                      & 0                                                                       \\ \hline
B & -150           & 100                      & -150                                                                     \\ \hline
C & 300           & 400                      & -150                                                                       \\ \hline
D & 200          & 600                      & -150                                                                    \\ \hline
E & -450          & 600                      & -600                                                                    \\ \hline
\end{tabular}
\end{table}

\begin{table}[ht]
\centering
\caption{}
\label{appa:modifiedexamplelog}
\begin{tabular}{|c|c|c|c|c|c|}
\hline
Stake & Match From & Match To & Remainder & Tx \# From & Tx \# To\\ \hline
-150  & 1          & 1        & -50      & 1 & 1 \\ \hline
300   & 1          & 1        & 250      & 2 & 2 \\ \hline
-450  & 2          & 3        & 0       &  3 & 4 \\ \hline
0  & 0         & 0        & 0 & 5 & 5            \\ \hline
\end{tabular}

\end{table}

So for example, in table \ref{appa:modifiedexamplelog}, the stake of -150 is matching against the stake at index 1 in the 'positive stake' list which is stake number 1 i.e. the stake from user A. The stake of 300 is matching against the stake at index 1 in the `negative stake' list, which is stake number 2 i.e. (what remains of) the stake from user B.
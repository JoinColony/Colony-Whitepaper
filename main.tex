\documentclass{article}
\usepackage[utf8]{inputenc}
\usepackage{geometry}
\usepackage{hyperref}
\title{Colony Whitepaper}
\author{}
\date{}
\def\code#1{\texttt{#1}}
%Define the listing package
\usepackage{listings} %code highlighter
\usepackage{color} %use color
\definecolor{mygreen}{rgb}{0,0.6,0}
\definecolor{mygray}{rgb}{0.5,0.5,0.5}
\definecolor{mymauve}{rgb}{0.58,0,0.82}
 
%Customize a bit the look
\lstset{ %
backgroundcolor=\color{white}, % choose the background color; you must add \usepackage{color} or \usepackage{xcolor}
basicstyle=\footnotesize, % the size of the fonts that are used for the code
breakatwhitespace=false, % sets if automatic breaks should only happen at whitespace
breaklines=true, % sets automatic line breaking
captionpos=b, % sets the caption-position to bottom
commentstyle=\color{mygreen}, % comment style
deletekeywords={...}, % if you want to delete keywords from the given language
escapeinside={\%*}{*)}, % if you want to add LaTeX within your code
extendedchars=true, % lets you use non-ASCII characters; for 8-bits encodings only, does not work with UTF-8
frame=single, % adds a frame around the code
keepspaces=true, % keeps spaces in text, useful for keeping indentation of code (possibly needs columns=flexible)
keywordstyle=\color{blue}, % keyword style
% language=Octave, % the language of the code
morekeywords={*,...}, % if you want to add more keywords to the set
numbers=left, % where to put the line-numbers; possible values are (none, left, right)
numbersep=5pt, % how far the line-numbers are from the code
numberstyle=\tiny\color{mygray}, % the style that is used for the line-numbers
rulecolor=\color{black}, % if not set, the frame-color may be changed on line-breaks within not-black text (e.g. comments (green here))
showspaces=false, % show spaces everywhere adding particular underscores; it overrides 'showstringspaces'
showstringspaces=false, % underline spaces within strings only
showtabs=false, % show tabs within strings adding particular underscores
stepnumber=1, % the step between two line-numbers. If it's 1, each line will be numbered
stringstyle=\color{mymauve}, % string literal style
tabsize=2, % sets default tabsize to 2 spaces
title=\lstname % show the filename of files included with \lstinputlisting; also try caption instead of title
}
%END of listing package%
 
\definecolor{darkgray}{rgb}{.4,.4,.4}
\definecolor{purple}{rgb}{0.65, 0.12, 0.82}
 
%define Javascript language
\lstdefinelanguage{JavaScript}{
keywords={typeof, new, true, false, catch, function, return, null, catch, switch, var, if, in, while, do, else, case, break},
keywordstyle=\color{blue}\bfseries,
ndkeywords={class, export, boolean, throw, implements, import, this},
ndkeywordstyle=\color{darkgray}\bfseries,
identifierstyle=\color{black},
sensitive=false,
comment=[l]{//},
morecomment=[s]{/*}{*/},
commentstyle=\color{purple}\ttfamily,
stringstyle=\color{red}\ttfamily,
morestring=[b]',
morestring=[b]"
}
 
\lstset{
language=JavaScript,
extendedchars=true,
basicstyle=\footnotesize\ttfamily,
showstringspaces=false,
showspaces=false,
numbers=left,
numberstyle=\footnotesize,
numbersep=9pt,
tabsize=2,
breaklines=true,
showtabs=false,
captionpos=b
}

\usepackage[T1]{fontenc}
\usepackage[usenames,dvipsnames]{xcolor}
\usepackage{soul}
\usepackage[breakable, theorems, skins]{tcolorbox}
\tcbset{enhanced}
\definecolor{light-gray}{gray}{0.92}
\sethlcolor{light-gray}

\newcommand{\ascode}[1]{\hl{\texttt{#1}}}

%% use \ascode{text} for inline code words
%% use \codebox{text} for blocks of code

\DeclareRobustCommand{\codebox}[1]{%
\begin{tcolorbox}[ %
        breakable,
        left=0pt,
        right=0pt,
        top=0pt,
        bottom=0pt,
        colback=light-gray,
        colframe=light-gray,
        width=\dimexpr\textwidth\relax, 
        enlarge left by=0mm,
        boxsep=5pt,
        arc=0pt,outer arc=0pt,
        ]
        \texttt{#1}
\end{tcolorbox}
}

\begin{document}

\maketitle
\setcounter{tocdepth}{2}
\tableofcontents

\section{Introduction}


\textbf{\emph{Insert some very general waffle-y stuff here containing such feel-good words as `coordination' and `cooperation'}}

\subsection{Overview}

In section \ref{sec:colonynetwork} we introduce the family of contracts that make up the \textbf{Colony Network}. This includes the Colony Network contract itself, the Colony Factory contract for creating new colonies, the storage contracts to manage data for a particular colony and the wallet contracts that manage the colony's tokens. This section also includes an outline of how upgrades to the contracts can be applied after the initial deployment.

Section \ref{sec:clny} introduces the \rcts\ (\textbf{CLNY}) that will underpin the colony network. The CLNY token is the token of the \textbf{\rc} --- a colony tasked with maintaining and developing the network. The CLNY token will be able to be used by all colonies, alongside Ether and their own native token, to pay for work. This section also describes how tokens obtained in the \textbf{Colony Crowdsale} can be converted into CLNY, and how we plan for control of the network to pass to the \rc\ over time.

To begin collaborating on a project, a user would send a transaction to the Colony Factory to deploy the contracts for a new colony. These contracts provide the governance mechanisms for users to make decisions collectively about the project. This includes decisions about what tasks need to be done, whether tasks should be funded, how much funding they get, and resolving disputes between members.

A Colony is a tool to coordinate effort to achieve an overall goal. We believe that dividing this goal into more achievable units will be essential for the success of any colony; we call these units \textbf{tasks}. Tasks can be assigned to users of the colony to be completed. The creation, assignment and completion of tasks is arguably the \emph{raison d'être} of a colony. Successful colonies are likely to have many tasks open at any one time; in order to ease the management of tasks, colonies can be divided into \textbf{domains}. These make it easy to group related tasks together and separate them from other unrelated tasks in other domains. Section \ref{sec:colony-structure} describes tasks and domains along with their internal structure and their hierarchy.

Completing a task entitles the user to claim the \textbf{bounty} assigned to the task, if any. Each colony is able to denominate bounties in its own token, in Ether, in CLNY, and other tokens allowed by the network that adhere to the ERC20 format \cite{erc20}. Bounties are assigned to tasks before workers are, with the bounty held in escrow by the colony to ensure the bounty can be claimed when the work is completed. The token allocation system of \textbf{Funding Proposals} is described in section \ref{sec:finance}. Tokens are assigned to domains and tasks on a continuous basis; the funding flows are directed by the users of the colony and are prioritised by the users' \textbf{reputation}. 

The reputation system is introduced in section \ref{sec:reputation}. Reputation is a key feature of the colony network, and is required to crate tasks and domains, as well as to fund them with tokens. Reputation is used to quantify the historical contributions of users to a colony, and to make sure they are justly rewarded. Reputation is not transferable between users, and slowly decays over time. This decay ensures that any reputation a user has is as a result of recent behaviour deemed beneficial to the colony as a whole. The calculations involved are too complex to carry out on the Ethereum blockchain. Updates to a user's reputation are calculated off-chain, with an on-chain reporting mechanism secured by economics and game theory. The details of this \textbf{Reputation Mining} process are the subject of section \ref{sec:reputationmining}.

Many decisions within a colony are made by consensus. Users are expected to monitor what their colleagues are doing, but hopefully will only rarely feel the need to intervene. Intervention in this context means `raising an objection' and is the subject of section \ref{sec:disputes}. Voting as a means of reaching decisions is enforced sparingly within Colony because it requires a lot of effort by many people and is slow. However, voting to resolve conflicts will be required in a colony. The \textbf{Dispute Resolution System} allows for any decision to be escalated to a vote of some or all users in the colony to resolve the disagreement. In order to make decision making close to meritocratic, we weigh ballots in such a vote by the users' relevant reputation.

Some colonies may be volunteer efforts or non-profit ventures, while others may be profit making businesses. Revenue generative colonies may vote to pay out a portion of this revenue to their token holders. When the colony pays out rewards, the reward a user receives is dependent on how many tokens they have and their reputation. The intention here is to ensure those who have contributed the most gain the greatest benefit. We hope that this incentivises users to keep contributing to a colony over the whole lifetime of the project, and not simply sit on accumulated tokens. The details of the \textbf{Reward Payout Process} are contained in section \ref{sec:revenue}.

Throughout this document, various numerical parameters are concretely specified. These values be subject to empirical review when the colony network begins live operation and any parameter values proposed in this document should be seen as good-faith suggestions, not prescriptions for the final network.

\section{The Colony Network on Ethereum}\label{sec:colonynetwork}

The Colony Network consists of a collection of contracts on the Ethereum blockchain. The contracts that must be deployed to instantiate the Colony Network are:

\begin{itemize}
\item The \code{ColonyNetworkResolver}
\item The \code{ColonyNetwork} contract
\item The \code{ColonyFactory}
\end{itemize}

The \code{ColonyFactory} contract deploys instances of the \code{Colony} contract and the ERC20 compliant contract that represents the colony's own token.

\subsection{Contract overviews}

\subsubsection{ColonyNetworkResolver contract}
A very simple, static contract that acts only as a pointer to the location of the ColonyNetwork Contract. The address of the \code{ColonyNetworkResolver} contract will never change, nor will we want it to. The address to which it points can be changed, initially by the Colony team, but ultimately only under the direction of the \code{ColonyNetwork} contract itself.

\subsubsection {ColonyNetwork contract}

The Colony Network hinges on the Colony Network Contract, which acts as the hub of the network. This contract is primarily responsible for managing the reputation mining process (described in section \ref{sec:reputationmining}), but also for general control of the network --- for example, setting the fees associated with using the network, or deploying a new version of the \code{ColonyFactory}.

\subsubsection {ColonyFactory contract}
This contract is used to deploy new instances of the \code{Colony} contract. These can either be new colonies, or part of the process of upgrading a colony.

\subsection{Contract upgradability}

We want to ensure the future upgradability of the deployed system as we foresee the Colony Network being continuously developed. Providing an upgrade path is important to allow people to use Colony without preventing themselves using new features as they are added to the Network.

While we had originally decided on one approach to provide this upgradability, using a storage contract we termed \ascode{EternalStorage}, the EIP150 hard fork increased the gas costs for the \ascode{call} opcode which this approach relied on heavily. As a result, we are reconsidering this decision. An explanation of the \ascode{EternalStorage} approach and of the alternative approach being currently considered can be found in Appendix \ref{appendix:upgradability}.

Regardless of the implementation used, we will ensure that in the case of a colony, the choice of upgrading the underlying \code{Colony} contract will never lie with the Colony Network. While the network is in control of what upgrades are available, they are not able to force an update upon a colony, and the colony itself must decide that it wants to upgrade to a new version.

\subsection{Contract security}\label{sec:escape-hatches}
While we aspire to bug free contracts, the adoption of a `defensive programming' mentality endeavours to limit the impact of any issues that manifest in the deployed contracts.

At launch, colonies will be able to be put into a `recovery mode'. In this state, whitelisted addresses are able to access functions that allow the state of the contract to be directly edited --- in practise, this will correspond to access to the functions to allow setting of variables, as well as being able to upgrade the contract. With the agreement of multiple whitelisted addresses, the contract will then be able to be taken out of recovery mode once the contract has been returned to a rational state. Removal from recovery mode requires the approval of multiple whitelisted addresses. This ensures that a single whitelisted address cannot, in a single transaction, enter recovery mode, make a malicious edit, and then exit recovery mode before the other parties on the whitelist have had a chance to react.

It is conceivable that colonies will be able to deactivate the recovery mode feature in the future, once the network and contracts have matured sufficiently.

In general, the contract may enter recovery mode due to:
\begin{itemize}
 \item A transaction from a whitelisted address signalling that the contract should enter recovery mode.
 \item Something that should always be true of the colony not being true --- for example, after a task payout checking that the amount of funds promised to tasks and not yet paid out is still less than the balance of the colony. If not, then abort the transaction and put the contract into recovery mode.
 \item A qualitative trigger suggesting something may be amiss --- perhaps too many tokens have been paid out in a short amount of time.
\end{itemize}

Any approvals from whitelisted addresses to leave recovery mode must be reset whenever a variable is edited. A whitelisted address agreeing to leave recovery mode records the timestamp at which the agreement occurred, and any change of variables also update a timestamp indicating the last edit. When attempting to leave recovery mode, only agreements made after the last edit are counted towards meeting the threshold.\footnote{We note that this is a loop only limited by the number of whitelisted addresses. An alternative implementation or a hard cap on the number of whitelisted addresses in each colony will therefore be required to ensure recovery mode can always be left.}

The first whitelisted address is added at colony creation and is the creator of the colony. Whitelisted addresses can be added or removed by a simple majority vote of existing whitelisted addresses. The Colony Foundation may be added as one of these whitelisted addresses as a `trusted party' if desired.
\section{Structure and Hierarchy within a Colony}
The purpose of colonies is to facilitate coordination among its members; to direct collective effort towards some goal(s). In light of this, one of the most central functions of a colony is to manage work assignments.

Work in a colony is divided up into well-defined jobs called `tasks', which can then be organised within a colony into `domains'. In this section, we describe each of these in detail and how we anticipate each to be used.

\subsection{Tasks}\label{sec:tasks}

The smallest structural unit in a colony is the `task'. A task represents a small unit of work that requires no further subdivision or delegation. A task has three roles associated with it:
\begin{itemize}
\item An administrator --- someone responsible for defining the task
\item A worker --- someone responsible for executing the task
\item An evaluator --- someone responsible for checking the work has been completed
\end{itemize}

These roles are assigned to addresses. While we anticipate that tasks will be mainly be defined such that each of these roles is assigned to an individual in the first instance, there is nothing preventing these addresses being contracts under the control of multiple people.\footnote{With the protocol described in this version of the document, any reputation earned would be assigned to the contract in question and not able to be moved to the appropriate users. We would expect some further developed version of the Colony Network to be able provide this functionality to users.}

The administrator (the creator of the task) is responsible for selecting the evaluator and worker (by whatever means they deem appropriate) and setting additional metadata for the task:

\begin{itemize}
\item A due date (represented by a block number)
\item Payouts for each of the administrator, the worker and the evaluator. We anticipate that the worker should be the role to get the bulk of each payout, as they are actually doing most of the work. 
\item A specification or brief for the task, to be used by the worker to guide the work, and by the evaluator for deciding if the work has been completed.
\end{itemize}

In order to create a task, the administrator must stake 0.01\% of the Colony Tokens in existence. This small stake is used to help discourage spamming of nonsense tasks, and provide a mechanism whereby the administrator can be punished upon bad behaviour. 

Defining what the payouts for each role should be, of course, does not provide the funds --- this must be done through the funding mechanisms in Colony (see Section \ref{sec:finance}). While the administrator is not the only person who is able to make the funding proposal --- which can be made by anyone --- they are the natural person to do so.

Assigning the worker account can only occur after the funds to satisfy the proposed payout has been received by the task, and both the administrator and proposed worker have accepted they are happy with the assignment. After that point, the worker account and payout cannot be changed, though other variables associated with the task can be.

Once all these elements have been set and funding is secured, the worker is able to --- at any point --- make a `final submission', which includes some proof that the work has been completed. This is a generic field that can be use for any proof that the work has been completed, but most likely should be an IPFS or Swarm hash referring to a longer file rather than a large proof that would be expensive to store on-chain.

Once the due date has passed or the worker has made their submission, the evaluator is able to grade the work. Regardless of whether the grading is positive or not, the task then enters a state where objections or `disputes' (see Section \ref{sec:disputes}) can be made over the final state of the task but no other changes can be made. Once three days have passed, no more disputes can be raised. Once all pending disputes related to the task are resolved, the parties involved get punished or rewarded based on the final state of the task.

If the evaluator's grading for the work is changed as a result of an objection (see Section \ref{sec:disputes}), they get a reduced payout based on the discrepancy between their original score and the score that their peers determined to be more appropriate. If an objection has determined the administrator should be punished, they lose their stake, otherwise the stake is returned. The worker is paid based on the final score awarded to their work, taking into account the result of any objections (or disputes).

\subsection{Domains}\label{sec:domains}

Of course, in a large colony, it would quickly become difficult for users to find relevant tasks just because of the sheer number of tasks. We therefore encourage users to create domains in their colony. These domains can contain both sub-domains and tasks. 

This compartmentalisation of activity in a colony provides a tangible benefit to the colony as a whole. When objections are raised, they can be raised to a specific level in the structural hierarchy within a colony. This means that people with relevant contextual knowledge can be targeted for their opinion, and also means that when a dispute occurs the whole colony does not have to vote in the dispute. Only those users with the relevant skills are asked for their opinion.

We also note that on some level, it is up to individual colonies to decide how they wish to use domains - some might only use them for coarse categorisations, whereas others may use them to very precisely group only the most similar tasks together, or even multiple tasks that other colonies would consider a single task. We aim to provide a general framework that colonies can use how they see fit, and to only be prescriptive in the service's use where necessary.


%schematic example diagram goes here



\begin{center}
 \begin{tikzpicture}
   \node[ellipse,draw, dashed] at (0,0) (tld) {Colony Domain};
   \node[ellipse,draw, dashed] at (-2,-2) (design) {Design};
   \node[ellipse,draw, dashed] at (2,-2) (development) {Development};
   \node[ellipse,draw, dashed] at (0.5,-4) (frontend) {Frontend};
   \node[ellipse,draw, dashed] at (4,-4) (backend) {Backend};
   \node[ellipse,draw, dashed] at (3,-6) (cpp) {C++};
   \node[ellipse,draw, dashed] at (5,-6) (golang) {Go}
    (tld.-120) edge[->, bend left=45, in=-120] (design.north)
    (tld.-60) edge[->, bend left=45, out=-60, in=120] (development.north)
    (development.-120) edge[->, bend left=45, in=-120] (frontend.north)
    (development.-60)  edge[->, bend left=45, out=-60, in=120] (backend.north)
    (backend.-120) edge[->, bend left=45, in=-120] (cpp.north)
    (backend.-60)  edge[->, bend left=45, out=-60, in=120] (golang.north);
  \node at (-2,-5.5) {\begin{tabular}{l} Parts of a domain hierarchy\\for a web development colony.\end{tabular}};
 \end{tikzpicture}
\end{center}

\section{The Reputation System}\label{sec:reputation}
\subsection{What is Reputation?}\label{subsec:what-is-reputation}

Within Colony, reputation is a number tied to an account. Reputation quantifies the contribution that user has made to the colony in recent history. Reputation will primarily be used for two things within a colony --- weighing votes of users, and determining rewards that should be allocated to users. The more reputation a user has, the more weight their vote has when making a decision. More reputation also means they receive a larger payout when rewards are issued (see Section \ref{sec:claimrewards}). We believe that because reputation is awarded to users by either direct or indirect peer approval of their actions, the consequences of this will be that influence and rewards in a colony will be assigned in meritocratic way.

\textbf{The Colony Governance System aims to be broadly meritocratic. For this reason, the majority of day-to-day decisions in a colony are weighted by the relevant reputation.} %Thus for example. any dispute in the development team will be resolved by a vote weighted by development reputation.

Unlike tokens, reputation cannot be transferred between accounts, and it cannot be bought or sold; it represents an appraisal of the account holder's activities by their peers. Reputation must therefore be earned by direct action within the colony. This reputation that is earned will eventually be lost through inaction, bad behaviour or being deemed to be wrong; an description of how reputation is gained and lost is given in Section \ref{sec:earning-losing-rep}. 

% Reputations held in one colony have no bearing on reputations held by the same account in another colony.

% Since it is not a tradable asset, reputation must be \emph{earned}. The primary method of earning reputation is by completing tasks in a colony and earning the colony's tokens. As soon as you successfully complete your first task in a colony and claim a reward of the colony's tokens, you also earn \emph{reputation} in that colony. The amount of reputation earned will be based on a combination of the performance of the user completing the task, and the payout associated with the task itself. Aside from tasks, there are administrative and managerial duties that allow users to earn reputation as well. (Section \ref{sec:earning-rep}).

% Any reputation earned is earned in a \emph{context}. The Colony Network tracks the domain in which reputation was earned and the skills associated with the action that earned it. As an example, reputation earned for the completion of a task in the `development' domain, tagged with the `solidity-dev' skill tag will be recorded as colony reputation as well as development reputation and solidity-dev reputation. (See Sections \ref{sec:rep-by-domain} and \ref{sec:rep-by-skill} for details).

% Reputation can also be lost. Any account engaged in bad behaviour will lose reputation as punishment. Furthermore \textbf{all reputation decays with time}. %Reputation has a half life of 3 months\footnote{The exact decay rate may change before release.}. 
% In order to maintain a high reputation score, an account must continue to contribute to the colony. (Section \ref{sec:losing-rep}).\\



%

%
%
\subsubsection{Reputation by Domain}\label{sec:rep-by-domain}
The organisational hierarchy of a colony provided by domains was described in Section \ref{sec:domains}. Reputation is earned in this hierarchy, and a user has a reputation in all domains that exist --- even if that reputation is zero. When a user earns or loses reputation in a domain, the reputation in all parent domains changes by the same amount. In the case of a user losing reputation, they also lose reputation in all child domains. Quantitatively, all child domains lose the same fraction of reputation that the user lost in the domain the reputation loss is notionally in. If ever a user would lose reputation and go below zero reputation, the relevant reputation becomes zero.

An example makes this clearer. Suppose a colony has a `development' domain which contains a `backend' domain and a `frontend' domain, as in Figure \ref{fig:domainhierarchysample}. Any time a member of the colony earns reputation for work completed in the backend domain, it will increase their backend reputation, their development reputation and their reputation in the all-encompassing top-level domain of the colony. Reputation earned in the development domain will only increase the development and top-level domain reputation scores of the user.

Later, the user behaves badly in the `development' domain, and they lose 100 reputation out of the 2000 they have in that domain. They also lose 100 reputation in the parent domains, and 5\% $\left(\frac{100}{2000}\right)$ of their reputation in each of the child domains of the `development' domain (which in this example, includes all of the Frontend, Backend, Node.js and Ruby domains). 

\subsubsection{Reputation by Skill}\label{sec:rep-by-skill}

We envision domains to mostly be used as an organisational hierarchy within a colony. However, this would not necessarily capture the \emph{type} of work that a user completed to earn their reputation. If the domain were a project, with tasks corresponding to both design and development work, reputation earned by completing tasks related to these skills would not be distinguishable.  To have a more fine-grained account of the type of work that a user completes to earn their reputation, the Colony Network also maintains a skill hierarchy for all colonies to use.

This global hierarchy of skills is available for all colonies to use. When a task is created, as well as being placed in a particular domain in the colony, it is also tagged with a skill from the skill hierarchy. When the worker earns reputation for successfully completing the task, they will earn reputation in the skill the task was tagged with and all parent skills. This is in addition to the reputation earned in the relevant domains. Conversely, if they are to lose reputation because their work is found inadequate, they will lose a proportional amount reputation from all child skills of the tag, if any, as is the case with the domain reputation. There is a top-level skill analogous to the top-level domain in a colony, which all skills are descendants of.

Even though the skill hierarchy is universal, reputation earned in the skill hierarchy is unique to each colony. Earning reputation in a skill in one colony has no effect on the user's reputation in that skill in any other colonies.

\subsubsection{Reputation by Colony}\label{sec:rep-by-colony}
A user's reputation in a colony is the sum of their reputation in the top-level skill and the top-level domain. This is the reputation they will be voting with in any decisions that require input from everyone in the colony. Reputation in a colony has no effect outside the colony.

\subsection{Earning and losing reputation}\label{sec:earning-losing-rep}
There are three ways to earn reputation in a colony. The first is being involved with a successfully completed task and the second is through the dispute process. In both of these cases, the user has been a productive member of the colony and is rewarded accordingly. The third way to earn reputation is upon the creation of a colony and the associated bootstrapping process (see Section \ref{sec:bootstrapping-rep}).

Reputation losses can arise from a user being found responsible for a badly executed task, or being involved in the dispute process and the dispute being resolved against them. In addition, all reputation earned by users is exposed to a continual decay over time. 

The rest of this section outlines each of these mechanisms, with references to the more detailed descriptions given elsewhere where appropriate.

\subsubsection{Reputation change from contributing to a task}\label{sec:earning-rep-from-task}
Each task requires three roles to be assigned: the administrator, the worker and the evaluator (as described in Section \ref{sec:tasks}). If the bounty for the task is denominated in the colony's token, each of these roles are eligible to earn reputation when the task is completed as long as their work was well received.

The performance of the user who has completed the work is established when the work is submitted and then evaluated. At this point, both the evaluator and the worker grade each other\footnote{These scores should be submitted using a pre-commit and reveal scheme to ensure secrecy during the rating process and avoid retaliatory grading} out of five stars.

In the case of the evaluator, a rating of 0-2 stars counts as them rejecting the work, and a score of 3-5 stars counts as accepting the work. Beyond that, we suggest the following guidelines for ratings:
\begin{itemize}
 \item[] 0 stars: user submitted no meaningful work 
 \item[]1 star:\phantom{s} user showed little activity relevant to the task, and remains far from completion on due date.
 \item[]2 stars: user was unable to complete the task, but put in a reasonable amount of effort.
 \item[]3 stars: user completed the task following the brief but there were issues during the work.
 \item[]4 stars: user completed the task acceptably and there were no complaints.
 \item[]5 stars: user completed the task to a higher standard than requested.
\end{itemize}

The actual number of reputation points $r$ earned by the worker for the completion of the task is then a function of this rating $s$ and the token payout $t$:
\begin{equation*}\label{eq:stars-to-rep}
 r = t \times \frac{2s - 5}{3}.
\end{equation*}
 
Reputation lost or gained as a function of the star rating therefore varies linearly between $-\frac{5t}{3}$ and $\frac{5t}{3}$ for zero and five stars respectively, and a rating of four stars earns the user exactly $t$. 

Similarly, the evaluator gets an amount of reputation based on their grading by the worker, but on a scale that only varies between $-t_{\rm ev}$ and $t_{\rm ev}$ (where $t_{\rm ev}$ is the evaluator's notional token payout for the task). They only earn this reputation in the current (and all parent) domains, not in the skill reputation hierarchy as they have not actually done the task. While it is likely some knowledge is required to perform the evaluation, this is not always the case; we believe that skill reputation should exclusively demonstrate ability to perform tasks.

Upon completion of a task, the administrator also earns reputation based on their token reward. There is no explicit rating of the administrator, but as with all other payments and rewards, an objection can be raised before and payout occurs. For all participants, reputation updates occur and payouts are made available only \emph{after} the objection window (described in Section \ref{sec:tasks}) has closed and all disputes  (described in Section \ref{sec:objections-and-disputes}) have been resolved at the end of the task. The reputation updates and payouts are based on the final state of the task.

\subsubsection{Reputation change as a result of Disputes}\label{sec:earning-rep-in-disputes}
If a dispute occurs, causing a vote among some portion of the colony, each side will have had to stake some number of tokens. Those who staked on the side determined to be right gain their stake back, plus some tokens that have been lost by the losing side. There will also be a reputation change as a result --- those on the losing side will lose some reputation, and some of that will be gained by the winning side. Section \ref{sec:objections-and-disputes} provides a full description of the dispute mechanism and the amount of tokens and reputation each side loses and gains.

\subsubsection{Bootstrapping reputation}\label{sec:bootstrapping-rep}
Since a colony's decision making procedure rests on reputation weighted voting, we are presented with a bootstrapping problem for new colonies. When a colony is new, no-one has yet completed any work in it and so nobody will have earned any reputation. Consequently, no objections can be raised and no disputes can be resolved as no-one is able to vote. Then, once the first task is successfully completed, that user has a dictatorship over decisions in the same domains or skills until another user earns similar types of reputation.

To prevent this, when a colony is created, the creator can choose addresses to have initial reputation assigned to them to allow the colony to bootstrap itself. There will be a global limit on the reputation that can be assigned in this manner in order to prevent an extreme reputation aristocracy. Given that reputation decays over time, this initial bootstrapping of reputation will not have an impact on the long-term operation of the colony. \textbf{This is the only time that reputation can be created without associated work being done.} Users receiving the reputation are presumably the colony creator or their acquaintances, and this starting reputation should be seen as a representation of the existing trust the creator has for their colleagues. 

We note that the same is not required when a new domain is created in a colony. We do not wish to allow the creation new reputation here, as this would devalue reputation already earned elsewhere in the colony. Happily, we can proceed without any new reputation. The colony is still able to make decisions and resolve disputes, because any objections can be escalated to a parent domain where reputation does exist, if necessary. Furthermore, even this escalation is not necessarily required in the event of a disagreement, because, even if there is no reputation in the \emph{domain} to contribute to the decision, users will still be able to vote based on their reputation in relevant \emph{skills}.

\subsubsection{Reputation Decay}
All reputation decays\footnote{The rate of decay in the final network may be different than specified here.} over time. Every 600000 blocks, a user's reputation in any domain or skill decays by a factor of 2. This decay occurs every hour, rather than being a step change every three months to ensure there are minimal incentives to earn reputation at any particular time. This frequent, network-wide update is the primary reason for the existence of the reputation mining protocol, which allows this near-continuous decay to be calculated off-chain without gas limits, and then realised on-chain. 

The decay serves multiple purposes. It ensures that reputation scores represent \emph{recent} contributions to a colony incentivising members to continually contribute to the colony. It further ensures that wild appreciations in token value (and the corresponding decrease in tokens paid per task) do not permanently distort the distribution of reputation but instead serves to smooth out the effects of such fluctuations over time.

\subsection{On-chain representation of skills and domains}\label{subsec:on-chain-representation-of-skills}
In the context of reputation, domains and skills are the same, differing only in that domains are colony-specific categorisation and skills are universal categorisation. In this subsection, each instance of `skill' should be taken to mean `skill or domain'.

Each skill that reputation can be earned in is assigned a \ascode{rep\_id} that is unique across the whole network. When a skill is created, additional properties are recorded and initialised.
\begin{equation*}
  \ascode{skill\_id} \rightarrow 
  \begin{cases}
    \ascode{n_parents} &	\textnormal{total number of parents}\\
    \ascode{parent\_n\_id} &	\parbox[t]{.6\linewidth}{\textnormal{the \ascode{rep\_id} of the $n^{\rm th}$ parent, where $n$ is an integer power of two larger than or equal to 1. }}\\
    \ascode{children}\left[\cdots\right] &	\textnormal{array of \ascode{rep\_id}s of all child skills}\\
    \ascode{n\_children} &	\textnormal{total number of child skills}
  \end{cases}
\end{equation*}
Upon creation, \ascode{children[]} and \ascode{n\_children} are empty. These two fields in all parents are updated with the \ascode{skill\_id} of the new skill on creation.\footnote{We acknowledge that this is fundamentally gas limited, but the only consequence of this will be the inability to create new skills once the maximum depth allowed by the block size is reached. Back-of-the-envelope calculations suggest this corresponds to a depth of around 80, which we don't believe our users will be limited by.}

Storing these pieces of data on-chain is required, as they are used by the reputation mining protocol (see next section) and the procedures for escalating disputes (see Section \ref{sec:objections-and-disputes}). They are stored under the control of the Colony Network contract.

\subsection{Reputation update log}\label{subsec:reputation-update-log}

Whenever an event that causes one or more users to have their reputation updated in a colony, a corresponding entry is recorded in a log in the Colony Network Contract. Each entry in the log contains

\begin{itemize}
\item The user suffering the reputation loss or gain.
\item The amount of reputation to be lost or gained.
\item The colony the update has occurred in.
\item How many reputation entries will need to be updated (including parent, child and colony-wide total reputations). This is the motivation for storing \ascode{generation} and \ascode{n\_children} for each skill and domain, as described in Section \ref{subsec:on-chain-representation-of-skills}.
\item How many total updates to reputations have occurred before this one in this cycle, including decays and updates to parents and children.
\end{itemize}

If the reputation update is the result of a dispute being resolved (as outlined in Section \ref{sec:earning-rep-in-disputes}), then instead of these first three properties, there is a reference to the dispute-specific record of stakes in the relevant colony. For the structure of this log, and an explanation of the way that it allows individual updates to be extracted in constant gas, see Appendix \ref{appendix:rep-transfer}.

This log exists to define an ordering of all reputation updates in a reputation update cycle that is accessible on-chain. In the event of a dispute during the reputation mining protocol (described in Section \ref{sec:reputationmining}), the Colony Network Contract can use this record to establish whether an update has been included correctly.


\section{Disputes and Arbitration}\label{sec:disputes}
%The what and why of the dispute system: permissive by default, dispute forces votes... FIXME.
Bureaucracies are slow and voting is cumbersome and takes time. Colony aims to be usable, efficient and fluid. The emphasis should be on `getting stuff done' and not about `applying for permission'. For this reason, Colony is designed to be \emph{permissive}. Explicitly, this means that task creation does not need explicit approval (Section \ref{sec:tasks}), neither does the process of getting funding for a task using a regular funding proposal (Section \ref{sec:finance}) nor any number of administrative actions throughout the colony system.\\
The assumption is that well aligned teams tend to be in agreement on most day-to-day goings on in their group. It is expected that members keep an eye out on what their colleagues are doing, but seldom feel the need to intervene. 

The \textbf{Dispute System} is there to resolve disagreements within the group and to punish bad behaviour and fraud. In short, the dispute mechanism allows colony members to signal disapproval and potentially \textbf{force a vote} on decisions and actions that would otherwise have proceeded unimpeded.


\subsubsection*{What are Objections?}
When a member of a colony feels that something is amiss, some task is overpriced, some evaluation incorrect, some funding proposal unjustified; they can \emph{raise an objection}. By doing so, they are fundamentally proposing that a variable, or more than one variable - stored in the \ascode{EternalStorage} contract should be changed to another value. For this reason we call supporters of the objection `the change side' and opponents `the keep side'.

The user raising the objection must also put up a stake to back it up (see Section \ref{sec:costs-of-disputes}). In essence, they are challenging the rest of the colony to disagree with them. In the spirit of avoiding uneccessary voting, the objection will pass automatically \emph{unless} someone else stakes on the counterside and thereby elevates the objection to a \emph{dispute}.

\subsubsection*{What are Disputes?}
We say that a dispute has been raised whenever an objection has found enough support on both the `change' side as well as the `keep' side. Once raised, disputes must be resolved by voting. 

\subsection{Objections and Disputes}\label{sec:objections-and-disputes}

The user raising an objection submits the following data:
\begin{itemize}
 \item the data that should be changed
 \item the reputation(s) that should vote on this issue (max. one each from organisational and skill hierarchy)
 \item proof that these reputations should be allowed to make the change in question. 
\end{itemize}

The first point is obvious - it is the subject of the objection. The second and third points concern \emph{escalation}. 

\begin{center}
 \textbf{In Colony you cannot escalate a decision to higher management, you can only escalate to bigger groups of your peers.}
\end{center}

For example, suppose that the objection concerns a task in the domain `development of our website'. The objection could chose to have all `development' reputation vote on it -- we say the decision was `escalated to the development domain'. In this example, the third point would be a proof that the domain `development of our website' was indeed a subdomain of `development'.

The highest domain any decision can be escalated to is the entire colony.

% 
% All variables in the EternalStorage contract are prefixed with what they relate to. For example, all variables to do with a task begin with \ascode{task\_}. For each type of variable, the governance system is programmed to know whether a permissions check is required. It seems likely it'll always be required, to me... can anyone come up with a variable that any group should be able to change on a whim?

%When a change is proposed, assuming that skill permission checks are required, we need to ensure that the reputation they are escalating to is a direct parent of the reputation associated with the variable being changed. This is possible to do efficiently because of metadata that is placed on the skills when they are created, which includes pointers to at least the direct parent of the skill (see the skill tree section of the reputation document). When a user creates a dispute, to specify the skill that they are escalating to they provide the lookups to be used from the skill(s) associated with the variable to be change, rather than directly specifying the skill they are escalating to. This ensures that the skill(s) they escalate to are direct parents of the skill associated with the variable.

\subsection{Costs and Rewards}\label{sec:costs-of-disputes}
\subsubsection{Cost of raising an objection}
To create an objection, a user must possess enough reputation and must also stake some number of the colony's tokens. How much reputation they need and how much they have to pay depends on the level they are escalating to; the `higher up' the decision goes, the higher the cost. To be considered a valid objection, the full requirement is for 1\% of the reputation queried and 1\% of the corresponding fraction of tokens to be staked. Thus, if an objection appeals to 13\% of total colony reputation, then the objection must be backed by 0.13\% (1\% of 13\%) of reputation and the required stake is 0.13\% of all colony tokens.\\
If the initial user does not have the required number of tokens or reputation, they can still create such a proposal by staking as little as 10\% of the reputation and tokens required.\footnote{This minimum amount required to even propose a change prevents users from spamming objections - even those that won’t ever be voted on - to large numbers of people, which would impede the smooth running of the colony.} In this case the objection will not be processed until other users add their support to it, taking it over the 1\% threshhold. 

\subsubsection{Cost of defending against a raised objection}
Once an objection has received sufficient backing it becomes active and, barring any further user actions for three days, the suggested change will take place. 
However, if there are users who oppose the suggested `change', they may add their support to the `keep' side. If the keep side receives sufficient support, a dispute is raised. 

If the `change' side does not garner enough support in three days, the objection fails and is rejected.
If, three days after the `change' side had enough tokens staked and the `keep' side does not, then it is assumed that the change is acceptable and it occurs.  %In either of these cases, the losing side loses (some of) their staked tokens and reputation.


%As staking occurs, metadata is saved to the blockchain along with each stake, to allow a gas-efficient method of inspecting a specific reputation transfer (required in the event of there being a dispute in the reputation state). There is a rough implementation in Python for those who wish to see how exactly it happens, but broadly speaking by keeping running totals for both sides staking as stakes are made and noting any partial matches that get made (e.g. first person stakes 100 on one side, the next person stakes 50 on the other, the first person has their stake partially unmatched), this allows for an arbitrary transfer to able to be referenced by index and not have to compute all previous transfers at that time (essentially, some of the calculation is done when the stake is made).  The number of reputation updates that need to be made (excluding the children and parent reputations) in such a situation will be equal to twice the number of stakers; some updates are `null' updates with no transfer to ensure this figure is always correct.

\subsubsection{Voting on Disputes}
If both sides stake the required number of tokens and reputation within their three days time limit, then the proposal goes to a vote.

The exact mechanisms of the vote are described elsewhere (Section \ref{sec:voting}). The weight of a user's vote is the sum of their reputations in the skills chosen by the user who originally raised the objection.


10\% of the staked tokens are set aside to pay voters when they vote; if a voter has 1\% of the reputation allowed to vote on a decision, they receive 1\% of this pot that is set aside. They receive this payout when they reveal their vote, regardless of the direction they voted in or the eventual result of the decision\footnote{\code{https://www.economicsnetwork.ac.uk/sites/default/files/Ashley/6\%20References\%20for\%20KBC.pdf}}. Any tokens assigned to users that do not vote in a poll are returned to the colony pot.


\subsubsection{Time to vote and quorum requirements}
The length of the voting period scales with the size of the reputation pool queried. A vote lasts between two days and seven days, where a vote would last two days if no reputation in the colony was being queried, and seven days in the case of the full colony being queried. At the end of this time frame, if quorum is not reached, no changes are made and all participants get their staked tokens returned. We define quorum to be more than 10\% of the reputation eligible to vote has done so.

\subsubsection{Consequences of the vote}
If quorum has been reached, if the `change' side won then the variable in question is changed, assuming that the reputation that voted for this outcome is more than previous vote on the same variable (see \ref{sec:repeated-disputes} below). If the `keep' side won, then the variable is not changed. In either case, the rest of the consequences are the same.

Alongside the variable that may or may not have been changed, the fraction of total reputation in the colony that voted for the winning side is noted. 

At the conclusion of the poll, losing stakers receive 0-90\% of their staked tokens back and the complementary percentage of the reputation they put at risk is lost. The exact amount of tokens they receive back (and therefore reputation they lose) is based on:

\begin{itemize}
 %\item The number of people that voted in a decision
 \item The fraction of the reputation in the colony that voted
 \item How close the vote ultimately was
\end{itemize}

At the end of a vote, if the vote was very close, then the losing side receives nearly 90\% of their stake back. If the vote is lopsided enough that the winning side's vote weight ($w$) reaches a landslide threshold ($L$) of the total vote weight, then they receive 0\% of their staked tokens back. $L$ varies based on the fraction of total reputation in the colony that was allowed to vote ($R$):

\[
L = 1 - \frac{R}{3}
\]

So for a small vote with little reputation in the colony being allowed to vote, the decision has to be close to unanimous for the losing side to be punished harshly. For a vote of the whole colony, the landslide threshhold $L$ reduces to 67\% of the votes - i.e. the reputation of the colony overall was split 2-to-1 on the decision.

Between these extremes of a landslide loss and a very slim loss, the loss of tokens and reputation suffered by the losing side ($\Delta$) varies linearly:

\[
 \Delta = 0.9 \times min \left\lbrace \frac{w-0.5}{L-0.5}, 1 \right\rbrace
\]


Any tokens lost beyond the initial 10\% are split between the colony and those who staked on the winning side, proportional to the amount they staked. Half of the reputation lost beyond the initial 10\% is given to those who staked on the winning side, and half is destroyed (the colony as a whole having reputation has no meaning, unlike the idea of the colony as a whole owning tokens).

The motivation behind this scheme is again one of efficiency. We aim to discourage spurious objections and disputes. We regard a close vote as a sign that the decision was indeed not a simple one and that forcing a vote on the issue may have been wise; on the other hand, if a vote ends in a landslide is a sign that the losing side was going up against a general consensus. We encourage communication within the colony. Members should be aware of the opinions of their peers whenever possible long before the dispute process is invoked.

\subsubsection*{Summary}
If you staked on the losing side of a dispute
\begin{itemize}
 \item 10\% of your stake is used to compensate voters for voting (unclaimed funds go to colony)
 \item Of the remaining 90\%, $\Delta$ is split between the opposing stakers and the colony
 \item $(90-\Delta)\%$ of your tokens are returned to you.
 \item You lose $(100-\Delta)\%$ of your staked reputation, with half of it going to opposing stakers (Section \ref{appendix:rep-transfer}).
\end{itemize}


\subsubsection{Repeated Disputes}\label{sec:repeated-disputes}
In order to prevent repeated objections and disputes over the same variable, the fraction of total reputation in the colony that voted for the winning side is recorded after every vote. This is the threshold that must be exceeded in a future vote in order to change the variable again. Note: this value is updated after every vote on the variable, even if the decision was to maintain the current value of the variable.



\subsection{Special case: Proposing an arbitrary transaction by the Colony contract}\label{sec:arbitrary-transaction}
It is desirable to have a mechanism by which a colony can create an arbitrary transaction on the blockchain to interact with contracts and tokens beyond those whitelisted by the network in advance. Such transactions should be rare occurances with high threshhold requirements.

Formally, proposing that a colony make an arbitrary transaction on the blockchain is no different from any other normal objection; the proposal is to set the value of a special variable to the value of the transaction data of the proposed transaction.\\
Such a proposal requires the entire colony to be able to vote (possibly both token holders and/or reputation holders), as the actions of the contract as a whole should be available for all to vote on. In the event the proposal is successful, the special variable is set. Another subsequent transaction - able to be made by anyone - is able to call a function that executes the transaction in the special variable, and resets it to empty if successful (to prevent it being called multiple times).

\subsection{Token-weighted, reputation-weighted and hybrid voting}
The majority of decisions in a colony are purely reputation weighted (even though creating a vote requires stake of both tokens and reputation), though there is no reason why a more traditional token-weighted vote shouldn't be available for some decisions, nor a hybrid vote based on both reputation and token holdings. In both of these cases, every account is allowed to vote, and in the case of a hybrid vote, all reputation is eligible to be leveraged. When a hybrid vote takes place, the total reputation and the total token holdings each represent 50\% of the voting weight.

The primary use of a token weighted vote is related to the management of the colony tokens itself; it seems reasonable that the decision and ability to create more tokens should lie with the colony token holders.

\section{CLNY Tokens, the Colony Network and the Common Colony}

\subsection{Creation of CLNY Tokens}
CLNY are generated during the Colony crowdsale period. Since the Colony Network will not yet be deployed at that time and thus not able to conduct a token sale internally; we will run the crowdsale using a widely-available crowdsale contract, which will be standalone.

After the Colony Network Contract is deployed, the first colony created will be the \rc. Tokens in this Colony will be CLNY. At this point, users will be able to `burn' the tokens purchased during the crowdsale 1:1 for tokens that are tokens in the \rc. In short, \rcts are the same as CLNY.

\subsection{Role of CLNY Holders}
CLNY holders, in the case of the fully realised Colony Network, have two primary roles. The first is management of the Colony Network itself. Holders vote on parameters that affect the network as a whole - for example, the fee charged by the network when a payment is made upon completion of a task (Section \ref{sec:tasks}). There will be permissioned functions on the Network Contract to allow these parameters to be set. 

In order for these permissioned \rc functions to be called by the colony contract, a vote must be taken where all CLNY holders and reputation holders are allowed to vote. Indeed, this restriction applies to all votes where the colony is making an arbitrary transaction (Section \ref{sec:arbitrary-transaction}).

Management of the Colony network also includes making updates to Colony contracts available to colonies. CLNY holders are not responsible for coding these updates, but the updates are only able to be deployed to the Colony Network with the assent of their assent, so they are responsible for at least doing due diligence to avoid introducing a security hole or other undesirable behaviour to the network. It is anticipated that actual work of this nature would take place within the \rc itself as a series of tasks.

The second primary role for the CLNY holders is the process we have termed `Reputation Mining' (Section \ref{sec:reputationmining}).

\subsection{Role of the \rc}
As alluded to above, the CLNY token is the token of the \rc. Like any other colony, the \rc has both tokens and reputation and its decision making is affected by both.\\
As a colony, the \rc is responsible for the development of the Colony Network i.e. the family of smart contracts that underpin the governance of and interaction between all colonies. In return, the \rc is the beneficiary of the network fee (Section \ref{sec:networkrevenue}).\\
Reputation in the \rc can be earned by earning CLNY tokens on tasks and administrative duties just as in any other colony (Section \ref{sec:earning-losing-rep}). Furthermore, reputation in the \rc can be earned by participating in the reputation mining process (Section \ref{subsec:mining-costs-and-rewards}).

\subsection{Handing off decision-making power to the \rc}\label{subsec:ceding-control-to-rc}
While CLNY holders are responsible from the start for Reputation Mining, to begin with, decisions about the functioning of the network, will be made by holders of a multisig key, owned by members of the Colony Foundation. As the network develops and is proved to be effective, control over these decisions will be released to the \rc, and the token holders within it.

\subsubsection*{Stage 1: Foundation Multisig in control}
Initially, the Network Contract's functions for making these changes will be permissioned to only allow transactions from the multisig address under the control of the Colony Foundation to change these properties of the network. 

\subsubsection*{Stage 2: Foundation Mutisig approval required}
At a later date, an intermediate contract will be set up, which will have these permissions given to it instead. This contract will be set up to allow the \rc (as a whole, via the governance mechanisms provided to all colonies) to propose changes to be made to the Colony Root Contract. The intermediate contract will have functionality such that all changes will have to be explicitly allowed by the address under the control of the Colony Foundation. In other words, the \rc will be able to propose changes, but the foundation must sign off on them.

\subsubsection*{Stage 3: Foundation Multisig retains veto}
The next stage will be a second intermediate contract which will operate very similarly to the first, but after a timeout - with no interaction from the Colony Foundation's address - the change will automatically be made to the Colony Root Contract. The Colony Foundation's role will be to block changes if necessary. The proposal to move to this contract will, of course, have to come from the \rc itself.  Thus at this stage the \rc will be able to make changes autonomously, but the foundation retains a veto.

\subsubsection*{Stage 4: \rc fuly in charge of the network}
Finally, the intermediate contract will be removed, and the \rc will have control over the Colony Root Contract with no direct control imbued to the Colony Foundation other than that provided by any \rct (CLNY) held by the Foundation. 

\subsection{Contract security \& escape hatches}
While it is extremely hard to make a contract with no flaws, it is slightly less hard to make a contract which is able to recover from flaws, or to minimise damage made by flaws.

At launch, colonies will be able to be put into a `recovery mode'. In this state, whitelisted addresses are able to access a function that directly allows the state of the contract to be edited - in practise, this probably corresponds to access to the functions in the StorageContract to allow editing of variables. With the agreement of multiple whitelisted addresses, the contract will then be able to be taken out of recovery mode once the contract has been returned to a rational state. We require multiple whitelisted addresses to approve the removal from recovery mode to ensure that a whitelisted address cannot, in a single transaction, enter recovery mode, make a malicious edit, and then exit recovery mode, making an edit to the contract state that would be difficult to otherwise detect.

In general, the contract may enter `recovery mode' due to:
\begin{itemize}
 \item A flag being raised by an appropriately permissioned user.
 \item Something that should always be true of the colony not being true - for example, amount of funds promised to tasks and not yet paid out should always be less than the balance of the colony.
 \item Some other trigger - perhaps too many tokens have been paid out in a short amount of time.
\end{itemize}

The Colony Foundation can be added as one of these whitelisted addresses as a `trusted party'. Note that if a Colony is able to be put into recovery mode, edited and taken out of recovery mode by the same user, this represents a security risk. It should be easier to enter recovery mode than to leave it.



\section{Finance: Managing Funds and Bounties}\label{sec:finance}
This section deals with the mechanisms by which a colony \emph{allocates} financial resources to domains and tasks. The norm is for resources to be allocated from the general to the particular, and that all allocations with sufficient reputational `backing' may proceed without a vote. As long as there is no disagreement, everything will run smoothly and automatically.

For how revenue earned by a colony is handled see Section \ref{sec:revenue}.

\subsection{Tokens and Ether}
Every colony has its own ERC20-compatible token. These tokens are under the control of the colony contract and may be used to pay for work done in the colony. Tokens only leave the control of the colony upon being paid out for completed tasks.\footnote{Currently, the only way this rule can be broken is by the Colony conspiring to abuse the Arbitrary Transaction feature described in Section \ref{sec:arbitrary-transaction}. }

To pay out tasks, in addition to local tokens, a colony may also use Ether, CLNY and other ERC20 tokens that have been explicitly whitelisted by the Colony Network.


\subsection{Funding pots and funding proposals}\label{sec:pots-and-fp}
All tokens and currencies are administered by the colony contract; it is responsible for all the bookkeeping and allocations.

\textbf{Each domain and each task in a colony has an associated \emph{funding pot}.} A funding pot can be thought of as acting like a wallet specific to a particular domain or task. To each funding pot, the colony contract may associate any number of unassigned tokens it holds. Depending on context, the funds in a funding pot may be referred to as the bounty, the budget, the salary or working capital. In addition to the funding pots, there is a special \emph{rewards pot} which accumulates tokens to be distributed to members as \textit{rewards} (see Section \ref{sec:revenue}).

\textbf{Funds are transferred between pots through \emph{funding proposals}.}
\begin{description}
 \item A Funding Proposal consists of the following data:
 \begin{itemize}
  \item \ascode{\textbf{Creator}}	--	The person that created the proposal.
  \item \ascode{\textbf{From}}	--	Funding pot funds are coming from.
  \item \ascode{\textbf{To}}	--	Pot funds are going to (may be the rewards pot).
  \item \ascode{\textbf{TokenType}}	--	The token contract address (0x0 for Ether).
  \item \ascode{\textbf{CurrentState}}	--	The state of the proposal (i.e. inactive, active, completed, cancelled).
  \item \ascode{\textbf{TotalPaid}}	--	How much has been transferred along this funding proposal so far.
  \item \ascode{\textbf{TotalRequested}}	--	The maximum amount to transfer after which this funding proposal is considered `completed'.
  \item \ascode{\textbf{LastUpdated}}	--	The time when the funding proposal was last updated.
  \item \ascode{\textbf{Rate}}	--	Rate of funding.
  \item \ascode{PunishCreator} -- Whether the creator should be punished upon completion.
 \end{itemize}

\end{description}
We distinguish between two types of funding proposals: Basic Funding Proposals (BFP) intended for normal use, and Priority Funding Proposals (PFP) intended to be used when atypical circumstances present themselves. The basic funding proposal may start funding the target straight away, whereas a priority funding proposal must be explicitly voted on before it starts directing funds. Furthermore, for a basic funding proposal the target pot must be a direct descendant of the source in the hierarchy whereas a priority funding proposal has no such restrictions.

Priority funding proposals should be used when funds need to be directed somewhere that is not a direct descendant of the source, when the funding rate needs to be very high (including immediate payment), or when the funding rate should be otherwise controlled (e.g. in the case of paying a salary).

For either funding proposal, the assignment to funding pots associated with domains or tasks is purely a bookkeeping mechanism. From the perspective of the blockchain, Ether and tokens are held by the colony contract until they are paid out when a task is completed.

\subsubsection{Creating a funding proposal}
Any member of the colony may create a funding proposal. The proposer must have 0.1\% of the reputation of the domain that is the most recent common ancestor of the source and target pots. They must stake an equivalent fraction of the colony's tokens. This stake is used to help discourage spamming of funding proposals and provide a mechanism whereby the creator can be punished for bad behaviour.

\subsubsection{From, To and TokenType}
The purpose of a funding proposal is to move tokens of \ascode{TokenType} from a pot \ascode{From} to a pot \ascode{To}.

The \ascode{TokenType} may be any ERC20 token whitelisted for use in the network, Ether, CLNY or the Colony's own Token. The \ascode{From} field must be a funding pot associated with a domain or a task in the colony, while the \ascode{To} field must be either a funding pot or the special rewards pot. If the funds are to move `downstream' from a domain to one of its children, a basic funding proposal is often sufficient.

\subsubsection{CurrentState}
The state of a funding proposal is either \ascode{inactive}, \ascode{active}, \ascode{completed} or \ascode{cancelled}. Only an active funding proposal is in line to channel funds. A basic funding proposal begins in active state while a priority one begins inactive (i.e. it must be activated by a vote). A funding proposal is completed when its \ascode{TotalPaid} reaches \ascode{TotalRequested}. Any other state changes must be made through the dispute mechanism (see Section \ref{sec:objections-and-disputes}).

\subsubsection{TotalPaid and TotalRequested}
The total number of funds that a funding proposal wishes to reallocate is called its \ascode{TotalRequested} amount. Due to the mechanism by which funding proposals accrue funds over time, it is common that a funding proposal will have received a part but not all of its \ascode{TotalRequested} amount. The total number of tokens accrued to date are stored in its \ascode{TotalPaid} amount.

\subsubsection{Rate and LastUpdated}
When a funding proposal is eligible to accrue funds (see Section \ref{subsec:funding-queue}) it does so at a specific \ascode{Rate}. Since nothing happens on the blockchain without user interaction, the funding system uses a form of lazy evaluation. To claim funds that the proposal is due, a user may `ping' the proposal --- i.e. the user manually requests an update. When pinged, the time since \ascode{LastUpdated} is multiplied by the \ascode{Rate} to determine how many tokens the proposal would have accrued in the interim if funding flow were continuous. This amount is added to \ascode{TotalPaid} and the current time is recorded as \ascode{LastUpdated}.

\ascode{TotalPaid} is only ever increased up to \ascode{TotalRequested} and when this happens as a result of a pinging transaction, the \ascode{LastUpdated} value is set to the earliest time at which this could have occurred.

\subsubsection{The funding queue}\label{subsec:funding-queue}
Active Funding Proposals that share the same \ascode{From} pot are ordered in a queue. At the top of the queue are the priority funding proposals, followed by the basic funding proposals. PFPs are ordered by the total reputation in their domain\footnote{The domain of a PFP is the domain that voted on it becoming active --- this will be the last common ancestor of the source and target pot domains unless an escalation has occurred.} --- while basic funding proposals are ordered by the reputation `backing' them.  The details of this procedure are outlined below.

%
%
%

\subsubsection{Basic funding proposals}\label{subsubsec:BFPs}
A basic funding proposal (\textbf{BFP}) is a funding proposal from some domain's funding pot to one of its children's. It starts out in the \ascode{active} state and is thus immediately eligible for funding. It may be cancelled at any time by the \ascode{Creator}, or through the dispute mechanism. In either case, the stake is only able to be reclaimed if \ascode{PunishCreator} has not been set to true via a dispute by the end of a timeout period.

\subsubsection{Ordering of BFPs}
Basic funding proposals are ordered in the \emph{Funding Queue}. Only one of them can receive funds at any one time. The proposals are ordered by the amount of reputation backing the proposal.

When created, a basic funding proposal gets placed at the bottom of the queue. Users can give a proposal `backing' weighted by their reputation in the source domain\footnote{The source domain of a BFP is the domain of the funding pot that the funding proposal is \ascode{From}.} at the time of backing\footnote{A user's reputation may change, but the backing weight is recorded at the time of backing and does not change without further user action.}. There are no costs to backing a proposal (other than gas costs) and the users obtain no direct benefits; it does not represent them putting their earned reputation at risk, nor any tokens --- it merely helps the proposal achieve funding in a more timely fashion.

The more reputation backs a proposal, the higher up the queue it is placed. Every transaction that adds backing to a proposal (or otherwise updates the backing level) inserts the proposal in the correct place in the queue. Only the funding proposal at the top of the queue accrues funds.

\subsubsection{The rate of funding for BFPs}
The more reputation backs a proposal, the faster it is funded. The rate scales linearly, and at the limit, if 100\% of the reputation in the source domain backs a basic funding proposal, then that funding proposal will be funded at a rate of 50\% of the domain's holdings (of the \ascode{TokenType}) per week. The goal is a steady and predictable allocation of resources directed collectively by the domain's (reputation weighted) priorities.

When a user backs a proposal, both the user and their reputation at the time are recorded. Consequently the user is able to update their backing at a later date. However, we note that such an update is not automatic and even if a user loses reputation due to bad behaviour, their backing level remains unchanged. To rectify this, we will allow users to update another user's backing to reflect their updated reputation scores, but we don't expect this functionality to be used often. We would only anticipate it being used if a user lost a lot of reputation due to some very bad behaviour, and other users wanted to prevent a bad funding proposal backed by the same user from being completed before it could be cancelled by other means (i.e. via dispute, described in Section \ref{sec:objections-and-disputes}).

We emphasised that a user could back a proposal with their reputation at the time of backing because the reputation backing a proposal will not change when that user's reputation does so. If by a quirk in this system, the reputation recorded as backing a funding proposal ends up higher than 100\% of the total of that reputation in the colony, then the funding occurs no quicker than it would at 100\%.

\subsubsection{Completing a BFP}
If an update finds that a proposal is fully funded (i.e. \ascode{TotalPaid} = \ascode{TotalRequested}), it is removed from this queue to allow the next-most-popular funding proposal to accrue funds. Explicitly, the following steps need to happen:
\begin{itemize}
 \item[\textbf{1.}] The time at which the funding proposal was fully funded is calculated.% (and is recorded as \ascode{LastUpdated})
 \item[\textbf{2.}] \ascode{TotalPaid} is set to \ascode{TotalRequested}.
 \item[\textbf{3.}] The BFP is removed from the queue.
 \item[\textbf{4.}] The next BFP in the queue is promoted to the top of the queue, and its \ascode{LastUpdated} time is set as the time calculated in \textbf{1.}
\end{itemize}

Once the the BFP has been fully funded, for a period of three days anyone can make a proposal that \ascode{PunishCreator} should be set to `true', and the creator lose their stake. This proposal takes the form of an Objection (Section \ref{sec:objections-and-disputes}). If no such proposal is made, or the proposal fails to pass, then the creator can reclaim their stake. Once the fate of the stake is decided, and the creator either reclaims the stake or loses it and a matching amount of reputation, the BFP is set to the \ascode{completed} state.


\subsubsection{Priority funding proposals}
A priority funding proposal (\textbf{PFP}) is a funding proposal that can request funds to be reallocated from any pot to any other at any rate. PFPs begin in the \ascode{inactive} state and can only become \ascode{active} via an explicit vote. The vote is based on reputation in the domain that is the most recent common ancestor of the two pots that money is being transferred between.

We imagine PFPs will be used to:
\begin{itemize}
 \item reclaim funds from child domains.
 \item reclaim funds from cancelled tasks.
 \item fund tasks across domains.
 \item set aside funds designated as a person's salary.
 \item make large, one-off payments.
 \end{itemize}


\subsubsection{PFPs and the funding queue}

Active Priority Funding Proposals take priority over Basic Funding Proposals and so they are placed at the top of the funding queue. They are ordered by the total reputation of the domain that voted to activate it and, in case there is a tie, by the actual amount of reputation that voted to activate. Thus PFPs that are higher in the domain hierarchy come before those lower down.

As with BFPs, any user can `ping' an active PFP at the top of the queue to cause the contract to update the funds available to the recipient pot. \ascode{TotalPaid}, \ascode{LastUpdated} and \ascode{CurrentState} are updated as required.

\subsubsection{The 24h waiting period for PFP updates}
Priority Funding Proposals take precedence over Basic Funding Proposals. To avoid the situation in which long running PFPs block the BFP process entirely, a limit is placed on how often and PFP can be updated (`pinged'). We say a PFP can only be pinged when it is first activated\footnote{In this initial update the time elapsed since last update is taken to be 24 hours.} and when its \ascode{LastUpdated} time is at least 24 hours old.

The result of this rule is that fast payments are still possible --- in such a case the PFP's \ascode{rate} is set very high and the proposal is fully funded at the initial ping, while also allowing long-term lower-rate PFPs that do not block the entire BFP process.

\subsubsection{When is a funding proposal eligible to receive funding?}
A Basic Funding Proposal may receive funds when pinged if it is active and at the top of the BFP funding queue and when the \ascode{LastUpdated} time of the PFPs are less than 24 hours old.

A Priority Funding Proposal may receive funds when pinged if it is active and all PFP ahead of it in the funding queue have been updated less than 24 hours ago.\footnote{In order to avoid hitting the gas limit due to unbounded loops, it will be necessary to maintain two orderings for the PFPs, one by priority and one by \ascode{LastUpdated}. }

\subsubsection{Editing funding proposals}
The creator of a funding proposal may edit the \code{TotalRequested} property of a funding proposal at any time, but doing so resets the reputational support that the proposal has in the funding queue to zero. The intention here is for changes to funding to be potentially quick to achieve with the agreement of others in the colony if the requirements for the recipient pot change (e.g. the scope of a domain increases).

\subsubsection{Cancelling funding proposals}
The \ascode{creator} of a funding proposal may set its \ascode{CurrentState} to \ascode{cancelled}. This is analogous to the creator of a task being able to cancel the task if it has not yet been assigned a worker (see Section \ref{sec:tasks}). Beyond this it must be possible for a colony to cancel a funding proposal without the creator's involvement. This is done through the objection mechanism described in Section \ref{sec:objections-and-disputes}.

When a task is cancelled, funding proposals that have that task's funding pot as their target (\ascode{To}) also enter the \ascode{Cancelled} state when they are next pinged, and no funds are reallocated. However, the funds that had already been transferred are not automatically returned; it will require a PFP to return the funds `upstream'.\footnote{It is conceivable that such return-funds-from-cancelled-tasks PFPs have lower hurdles of activation.}

\subsection{Paying out a task bounty}\label{sec:claiming-bounty}
Ultimately, via the mechanism described above, some tokens have found their way into a funding pot associated with a task. Upon completion of the specified work and approval by the evaluator, the worker has earned the contents of the funding pot. However, even once the work has been approved, there remains a period of time before the funds can be requested to be paid out by the receiving address. This is enforced to allow a period where a user of the colony can object to the payout, accusing it of fraudulence or similar.

If a user objects to the payout, they can raise an objection to void the task payout. The objection may be raised to a parent domain of the task in question --- or even escalated to the colony domain itself. The choice of domain lies with the user making the objection. Users vote to resolve any resulting disputes with votes weighted by their reputation in the domain the objection was raised to. In the event that their dispute is upheld, the funds can be returned to the domain that contained the task. If the voters approve of the payout, no changes are made and the user will be able to claim their payout after all.

While a payout is under dispute, the timeout period continues to run. After the end of the dispute period, no more objections are able to be raised, but any existing objections are resolved before payout is allowed.  This is to prevent users being able to continually raise disputes to prevent payout.

Once the tokens have been paid out, they are under the control of the user --- there is no way to reclaim the funds. The funds have to cross the `Cryptographic Rubicon' somewhere in the system (by the nature of the blockchain), and it makes sense to do so here.


\end{document}
